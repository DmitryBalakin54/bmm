\documentclass[a4paper,14pt,russian]{extreport}
\usepackage{extsizes}
\usepackage{cmap}
\usepackage{amsmath}
\usepackage{tabularx}
\usepackage{eufrak}
\usepackage[T2A]{fontenc} % Поддержка кириллицы
\usepackage[utf8]{inputenc} % Кодировка UTF-8
\usepackage[russian]{babel} % Поддержка русского языка
%\usepackage{mathptmx}
\setlength{\parindent}{1.25cm} % установка абзацного отступа
\usepackage{fontspec} % Для использования системных шрифтов
\setmainfont{Times New Roman}

% Добавляем эти строки для управления нумерацией
\usepackage{etoolbox}
\AtBeginDocument{\addtocounter{page}{-1}} % Начинаем нумерацию с титульника
\pretocmd{\titlepage}{\thispagestyle{empty}}{}{}


\addto\captionsrussian{%
  \renewcommand{\contentsname}{Содержание}%
}


\usepackage{graphicx} % Required for inserting images
\graphicspath{{images/}}
\usepackage{amsfonts}
\usepackage{indentfirst} % отделять первую строку раздела абзацным отступом тоже
\usepackage{amssymb} % Для символа \varnothing
\usepackage{makecell}
\linespread{1.3} % полуторный интервал
%\renewcommand{\rmdefault}{ftm} % Times New Roman
\frenchspacing
\usepackage{fancyhdr}
\fancypagestyle{plain}{
    \fancyhf{} % Очищаем все
    \fancyhead[C]{\thepage} % Номер страницы по центру вверху
    \renewcommand{\headrulewidth}{0pt} % Убираем горизонтальную линию
}
\usepackage{titlesec}
\usepackage{setspace} % Для управления межстрочным интервалом





%--figures imports--
\usepackage{tikz}
\usetikzlibrary{calc, angles}
\usetikzlibrary{arrows.meta}

% \usepackage{caption}

\newcommand{\capt}[1]{%
    \\[8pt]
    \text{Рис. #1}
}

\usepackage{hyperref}
\hypersetup{
    colorlinks=true,
    linkcolor=black,
    filecolor=magenta,      
    urlcolor=cyan,
    pdftitle={Overleaf Example},
    pdfpagemode=FullScreen,
    }
    
% Общие настройки для всех рисунков
\tikzset{
    my drawing/.style={
       scale=#1,
      line join=round,
      line cap=round,
      thick/.style={line width=0.15pt * #1},
      dashed/.style={dash pattern=on 2pt * #1 off 1pt * #1, line width=0.15pt * #1},
      vertex/.style={fill=black, circle, inner sep=0.25pt * #1},
     % my label/.style={font=\footnotesize},
     my label/.style={
    font=\fontsize{10pt}{25pt}\selectfont,
    inner sep=0.2pt,
    outer sep=1.8pt,
    scale=0.3 * #1
},
        my formula/.style={
            font=\fontsize{12.5pt}{25pt}\selectfont,
            inner sep=0.2pt,
            outer sep=4pt,
            scale=0.2 * #1,
        },
      rotate around z=0,
      rotate around x=0,
      rotate around y=0,
    }
}



\newcommand{\drawcube}[2][]{
  \begin{tikzpicture}[my drawing=#2,#1]
    
    \def\x{1.0}
    \def\y{0.5}
    \def\z{1.2}
    
    \coordinate (A) at (0,0);
    \coordinate (B) at (\x,0);
    \coordinate (C) at (\x+\y,\y);
    \coordinate (D) at (\y,\y);
    \coordinate (A1) at (0,\z);
    \coordinate (B1) at (\x,\z);
    \coordinate (C1) at (\x+\y,\y+\z);
    \coordinate (D1) at (\y,\y+\z);

    \draw[dashed] (A) -- (D);
    \draw[dashed] (D) -- (C);
    \draw[dashed] (D) -- (D1);

    \draw[thick] (A) -- (B) -- (C);
    \draw[thick] (B) -- (B1) -- (C1) -- (C);
    \draw[thick] (B1) -- (A1);
    \draw[thick] (C1) -- (D1);
    \draw[thick] (B) -- (B1);
    \draw[thick] (A1) -- (D1);
    \draw[thick] (A) -- (A1);

    \filldraw[fill=blue!15, draw=blue!80, thick, opacity=0.7] 
      (A) -- (B) -- (C1) -- (D1) -- cycle;

    \node[below left, my label] at (A) {$A$};
    \node[below right, my label] at (B) {$B$};
    \node[above right, my label] at (C) {$C$};
    \node[left, my label] at (D) {$D$};
    \node[left, my label] at (A1) {$A_1$};
    \node[right, my label] at (B1) {$B_1$};
    \node[right, my label] at (C1) {$C_1$};
    \node[left, my label] at (D1) {$D_1$};

    \foreach \point in {A,B,C,D,A1,B1,C1,D1} {
      \node[vertex] at (\point) {};
    }
  \end{tikzpicture}
}


% \newcommand{\drawCubeTaskOne}[2][]{
%   \begin{tikzpicture}[
%       scale=#2,
%       line join=round,
%       line cap=round,
%       thick/.style={line width=0.3pt},
%       dashed/.style={dash pattern=on 2pt off 1pt, line width=0.3pt},
%       vertex/.style={fill=black, circle, inner sep=0.5pt},
%       my label/.style={font=\footnotesize},
%       rotate around z=0,
%       rotate around x=0,
%       rotate around y=0,
%       #1
%     ]
    
%     \def\x{1.0}
%     \def\y{1.0}
%     \def\z{1.0}
    
%     % \coordinate (A) at (0,0);
%     % \coordinate (D) at (\x,0);
%     % \coordinate (C) at (\x * 4 / 11 + \x,\x * 7 / 11);
%     % \coordinate (B) at (\x * 4 / 11,\x * 7 / 11);
%     % \coordinate (A1) at (0,\x);
%     % \coordinate (D1) at (\x,\x);
%     % \coordinate (C1) at (\x * 4 / 11 + \x,\x * 7 / 11 + \x);
%     % \coordinate (B1) at (\x * 4 / 11,\x * 7 / 11 + \x);

    
%     % \coordinate (A) at (0,0);
%     % \coordinate (D) at (\x * 0.8660254038, \x * (-0.5));
%     % \coordinate (C) at (1, (\x * 7 / 11) * 0.8660254038 + (\x * 4 / 11 + \x) * (-0.5));
%     % \coordinate (B) at (\x * 4 / 11,\x * 7 / 11);
%     % \coordinate (A1) at (0,\x);
%     % \coordinate (D1) at (\x,\x);
%     % \coordinate (C1) at (\x * 4 / 11 + \x,\x * 7 / 11 + \x);
%     % \coordinate (B1) at (\x * 4 / 11,\x * 7 / 11 + \x);

%     \coordinate (A) at (0,0);
%     \coordinate (D) at (\x,0);
%     \coordinate (C) at (\x * 4 / 11 * 0.9 + \x,\x * 7 / 11 * 0.9);
%     \coordinate (B) at (\x * 4 / 11 * 0.9,\x * 7 / 11 * 0.9);
%     \coordinate (A1) at (0,\x);
%     \coordinate (D1) at (\x,\x);
%     \coordinate (C1) at (\x * 4 / 11 * 0.9 + \x,(\x * 7 / 11 + \x) * 0.9);
%     \coordinate (B1) at (\x * 4 / 11 * 0.9,(\x * 7 / 11 + \x) * 0.9);

%     \coordinate (K) at ($(A)!1/3!(C1)$);
%     \coordinate (O) at ($(A)!1/2!(C)$)
    

%     \draw[dashed] (A) -- (B);
%     \draw[dashed] (B) -- (C);
%     \draw[dashed] (B) -- (B1);
%     \draw[dashed] (A) -- (C1);
%     \draw[dashed] (A) -- (C);
%     \draw[dashed, blue] (B) -- (D);
%     \draw[dashed] (O) -- (A1);
%     \draw[dashed, blue] (A1) -- (B);

%     \draw[thick] (A) -- (D) -- (C);
%     \draw[thick] (D) -- (D1) -- (C1) -- (C);
%     \draw[thick] (D1) -- (A1);
%     \draw[thick] (C1) -- (B1);
%     \draw[thick] (D) -- (D1);
%     \draw[thick] (A1) -- (B1);
%     \draw[thick] (A) -- (A1);
%     \draw[thick, blue] (D) -- (A1);
    

%     % \filldraw[fill=blue!15, draw=blue!80, thick, opacity=0.7] 
%     %   (A) -- (B) -- (C1) -- (D1) -- cycle;

%     \node[below left, my label] at (A) {$A$};
%     \node[below right, my label] at (D) {$D$};
%     \node[above right, my label] at (C) {$C$};
%     \node[left, my label] at (B) {$B$};
%     \node[left, my label] at (A1) {$A_1$};
%     \node[right, my label] at (D1) {$D_1$};
%     \node[right, my label] at (C1) {$C_1$};
%     \node[left, my label] at (B1) {$B_1$};
%     \node[right, my label] at (K) {$K$};
%     \node[left, my label] at (O) {$O$};

%     \foreach \point in {A,B,C,D,A1,B1,C1,D1,K,O} {
%       \node[vertex] at (\point) {};
%     }
%   \end{tikzpicture}
% }




\newcommand{\drawCubeTaskOne}[2][]{
  \begin{tikzpicture}[my drawing=#2,#1]
    
    \def\x{1.0}
    \def\y{1.0}
    \def\z{1.0}
    
    % \coordinate (A) at (0,0);
    % \coordinate (D) at (\x,0);
    % \coordinate (C) at (\x * 4 / 11 + \x,\x * 7 / 11);
    % \coordinate (B) at (\x * 4 / 11,\x * 7 / 11);
    % \coordinate (A1) at (0,\x);
    % \coordinate (D1) at (\x,\x);
    % \coordinate (C1) at (\x * 4 / 11 + \x,\x * 7 / 11 + \x);
    % \coordinate (B1) at (\x * 4 / 11,\x * 7 / 11 + \x);

    
    % \coordinate (A) at (0,0);
    % \coordinate (D) at (\x * 0.8660254038, \x * (-0.5));
    % \coordinate (C) at (1, (\x * 7 / 11) * 0.8660254038 + (\x * 4 / 11 + \x) * (-0.5));
    % \coordinate (B) at (\x * 4 / 11,\x * 7 / 11);
    % \coordinate (A1) at (0,\x);
    % \coordinate (D1) at (\x,\x);
    % \coordinate (C1) at (\x * 4 / 11 + \x,\x * 7 / 11 + \x);
    % \coordinate (B1) at (\x * 4 / 11,\x * 7 / 11 + \x);

    \coordinate (A) at (0,0);
    \coordinate (D) at (\x,0);
    \coordinate (C) at (\x * 4 / 11 * 0.9 + \x,\x * 7 / 11 * 0.3);
    \coordinate (B) at (\x * 4 / 11 * 0.9,\x * 7 / 11 * 0.3);
    \coordinate (A1) at (0,\x);
    \coordinate (D1) at (\x,\x);
    \coordinate (C1) at (\x * 4 / 11 * 0.9 + \x, 1.19);
    \coordinate (B1) at (\x * 4 / 11 * 0.9, 1.19);

    \coordinate (K) at ($(A)!1/3!(C1)$);
    \coordinate (O) at ($(A)!1/2!(C)$);
    \coordinate (F) at ($(C1)!1/2!(C)$);


    \draw[dashed] (A) -- (B);
    \draw[dashed] (B) -- (C);
    \draw[dashed] (B) -- (B1);
    \draw[dashed] (A) -- (C1);
    \draw[dashed] (A) -- (C);
    
    
    \draw[dashed] (O) -- (F);

    \draw[thick] (A) -- (D) -- (C);
    \draw[thick] (D) -- (D1) -- (C1) -- (C);
    \draw[thick] (D1) -- (A1);
    \draw[thick] (C1) -- (B1);
    \draw[thick] (D) -- (D1);
    \draw[thick] (A1) -- (B1);
    \draw[thick] (A) -- (A1);

    \filldraw[fill=blue!15, draw=blue!0, thick, opacity=0.7] 
        (A1) -- (B) -- (D) -- cycle;
    

    \draw[dashed, blue] (A1) -- (B);
    \draw[thick, blue] (D) -- (A1);
    \draw[dashed, blue] (B) -- (D);


    \draw[dashed] (O) -- (A1);
    \draw[dashed] (A) -- (K);
    


    \pic [draw, thick, angle radius=2mm, angle eccentricity=1.2] {right angle = A--O--B};
    \pic [draw, thick, angle radius=2mm, angle eccentricity=1.2] {right angle = F--O--D};
    
    

   

    \node[below left, my label] at (A) {$A$};
    \node[below right, my label] at (D) {$D$};
    \node[above right, my label] at (C) {$C$};
    \node[left, my label] at (B) {$B$};
    \node[left, my label] at (A1) {$A_1$};
    \node[above left, my label] at (D1) {$D_1$};
    \node[right, my label] at (C1) {$C_1$};
    \node[above left, my label] at (B1) {$B_1$};
    \node[right, my label] at (K) {$K$};
    \node[below, my label] at (O) {$O$};
    \node[right, my label] at (F) {$F$};
    

    \foreach \point in {A,B,C,D,A1,B1,C1,D1,K,O,F} {
      \node[vertex] at (\point) {};
    }
  \end{tikzpicture}
}


\newcommand{\drawCubeTaskOnePTwo}[2][]{
  \begin{tikzpicture}[my drawing=#2,#1]
    
    \def\x{1.0}
    \def\y{1.0}
    \def\z{1.0}


    \coordinate (A) at (0,0);
    \coordinate (C) at (1.41 * \x, 0);
    \coordinate (A1) at (0,\x);
    \coordinate (C1) at (1.41 * \x, \x);

    \coordinate (K) at ($(A)!1/3!(C1)$);
    \coordinate (O) at ($(A)!1/2!(C)$);
    

    \draw[thick] (A) -- (C1);
    \draw[thick] (A) -- (C);
    \draw[thick] (A) -- (A1);
    \draw[thick] (C) -- (C1);
    \draw[thick] (C1) -- (A1);
    \draw[thick] (O) -- (K);
    

    \pic [draw, thick, angle radius=2mm, angle eccentricity=1.2] {right angle = A--K--O};
    \pic [draw, thick, angle radius=2mm, angle eccentricity=1.2] {right angle = O--C--C1};
    \pic [draw, thick, angle radius=3mm] {angle=O--A--K};


    \node[below left, my label] at (A) {$A$};
    \node[below right, my label] at (C) {$C$};
    \node[left, my label] at (A1) {$A_1$};
    \node[right, my label] at (C1) {$C_1$};
    \node[right, my label] at (K) {$K$};
    \node[below, my label] at (O) {$O$};

    \draw (A) -- (O) node[midway, below, sloped, my formula] {$\frac{a \sqrt{2}}{2}$};
    \draw (C) -- (O) node[midway, below, sloped, my formula] {$\frac{a \sqrt{2}}{2}$};
    \draw (A) -- (K) node[midway, above, sloped, my formula] {$\frac{a \sqrt{3}}{3}$};
    \draw (C1) -- (K) node[midway, above, sloped, my formula] {$\frac{2 a \sqrt{3}}{3}$};
    \draw (C) -- (C1) node[midway, right, my formula] {$a$};



    \foreach \point in {A,C,A1,C1,K,O} {
      \node[vertex] at (\point) {};
    }
  \end{tikzpicture}
}


\newcommand{\drawCubeTaskOnePThree}[2][]{
  \begin{tikzpicture}[my drawing=#2,#1]
    
    \def\x{1.0}
    \def\y{1.0}
    \def\z{1.0}


    \coordinate (A) at (0,0);
    \coordinate (C) at (1.41 * \x, 0);
    \coordinate (A1) at (0,\x - 0.2);
    \coordinate (C1) at (1.41 * \x, \x - 0.2);
    \coordinate (M) at (0.15 - 0.05, \x - 0.2);

    \coordinate (K) at ($(A)!105/300!(C1)$);
    \coordinate (O) at ($(A)!100/200!(C)$);
    

    \draw[thick] (A) -- (C1);
    \draw[thick] (A) -- (C);
    \draw[thick] (A) -- (A1);
    \draw[thick] (C) -- (C1);
    \draw[thick] (C1) -- (A1);
    \draw[thick] (O) -- (K);
    \draw[thick] (O) -- (M);
    

    \pic [draw, thick, angle radius=2mm, angle eccentricity=1.2] {right angle = A--K--O};
    \pic [draw, thick, angle radius=2mm, angle eccentricity=1.2] {right angle = O--C--C1};
    \pic [draw, thick, angle radius=3mm] {angle=O--A--K};
    
    \pic [draw, thick, angle radius=3mm] {angle=K--M--C1};
    \pic [draw, thick, angle radius=2mm] {angle=K--M--C1};
    \pic [draw, thick, angle radius=2mm, angle eccentricity=1.2] {right angle = M--K--C1};
    \pic [draw, thick, angle radius=2mm] {angle=K--O--A};
    \pic [draw, thick, angle radius=3mm] {angle=K--O--A};
    \pic [draw, thick, angle radius=3mm] {angle=M--C1--K};


    \node[below left, my label] at (A) {$A$};
    \node[below right, my label] at (C) {$C$};
    \node[left, my label] at (A1) {$A_1$};
    \node[right, my label] at (C1) {$C_1$};
    \node[right, my label] at (K) {$K$};
    \node[below, my label] at (O) {$O$};
    \node[above, my label] at (M) {$M$};

    \draw (A) -- (O) node[midway, below, sloped, my formula] {$\frac{a \sqrt{2}}{2}$};
    \draw (C) -- (O) node[midway, below, sloped, my formula] {$\frac{a \sqrt{2}}{2}$};
    \draw (A) -- (K) node[midway, above, sloped, my formula] {$\frac{a \sqrt{3}}{3}$};
    \draw (C1) -- (K) node[midway, above, sloped, my formula] {$\frac{2 a \sqrt{3}}{3}$};
    \draw (C) -- (C1) node[midway, right, my formula] {$a$};



    \foreach \point in {A,C,A1,C1,K,O, M} {
      \node[vertex] at (\point) {};
    }
  \end{tikzpicture}
}





\newcommand{\drawPyramidTask}[2][]{
  \begin{tikzpicture}[my drawing=#2,#1]
    
    \def\x{1.0}
    \def\y{1.0}
    \def\z{1.0}

    \coordinate (A) at (0,0);
    \coordinate (D) at (\x,0);
    \coordinate (C) at (\x * 3 / 8 + \x,\x * 1 / 2);
    \coordinate (B) at (\x * 3 / 8,\x * 1 / 2);

    \coordinate (O) at ($(A)!1/2!(C)$);
    \coordinate (S) at ($(O) + (0, 1.5 * \x)$);

    \coordinate (M) at ($(S)!7/10!(C)$);
    \coordinate (N) at ($(D)!9/10!(C)$);
    

    \draw[dashed] (A) -- (B);
    \draw[dashed] (B) -- (C);
    \draw[dashed] (A) -- (C);
    \draw[dashed] (B) -- (D);
    \draw[dashed] (B) -- (S);
    \draw[dashed] (S) -- (O);
    
    

    \draw[thick] (A) -- (D) -- (C);
    \draw[thick] (A) -- (S);
    \draw[thick] (D) -- (S);
    \draw[thick] (A) -- (S);
    \draw[thick] (C) -- (S);
    
    

    \node[below left, my label] at (A) {$A$};
    \node[below right, my label] at (D) {$D$};
    \node[above right, my label] at (C) {$C$};
    \node[left, my label] at (B) {$B$};
    \node[below, my label] at (O) {$O$};
    \node[above, my label] at (S) {$S$};
    \node[above right, my label] at (M) {$M$};
    \node[below right, my label] at (N) {$N$};

    \foreach \point in {A,B,C,D,O,S,M,N} {
      \node[vertex] at (\point) {};
    }
  \end{tikzpicture}
}

\newcommand{\drawPyramidTaskPTwo}[2][]{
  \begin{tikzpicture}[my drawing=#2,#1]
    
    \def\x{1.0}
    \def\y{1.0}
    \def\z{1.0}

    \coordinate (A) at (0,0);
    \coordinate (D) at (\x,0);
    \coordinate (C) at (\x * 3 / 8 + \x,\x * 1 / 2);
    \coordinate (B) at (\x * 3 / 8,\x * 1 / 2);

    \coordinate (O) at ($(A)!1/2!(C)$);
    \coordinate (S) at ($(O) + (0, 1.5 * \x)$);

    \coordinate (M) at ($(S)!7/10!(C)$);
    \coordinate (N) at ($(D)!9/10!(C)$);
    \coordinate (K) at ($(A)!7/10!(C)$);
    \coordinate (T) at ($(B)!23/30!(A)$);
    \coordinate (F) at ($(B)!23/30!(S)$);
    
    

    \draw[dashed] (A) -- (B);
    \draw[dashed] (B) -- (C);
    \draw[dashed] (A) -- (C);
    \draw[dashed] (B) -- (D);
    \draw[dashed] (B) -- (S);
    
    \draw[thick] (A) -- (D) -- (C);
    \draw[thick] (A) -- (S);
    \draw[thick] (C) -- (S);


    \filldraw[fill=blue!15, draw=blue!0, thick, opacity=0.7] 
            (N) -- (M) -- (F) -- (T) -- cycle;

    \draw[dashed, blue] (K) -- (M);
    \draw[dashed, blue] (T) -- (N);
    \draw[dashed, blue] (T) -- (F);
    \draw[dashed, blue] (M) -- (F);
    
    \draw[thick, blue] (M) -- (N);

    \draw[thick] (A) -- (S);
    \draw[thick] (D) -- (S);
    \draw[dashed] (S) -- (O);
    
    
    
    
    

    \node[below left, my label] at (A) {$A$};
    \node[below right, my label] at (D) {$D$};
    \node[above right, my label] at (C) {$C$};
    \node[left, my label] at (B) {$B$};
    \node[below, my label] at (O) {$O$};
    \node[above, my label] at (S) {$S$};
    \node[above right, my label] at (M) {$M$};
    \node[below right, my label] at (N) {$N$};
    \node[above, my label] at (K) {$K$};
    \node[right, my label] at (T) {$T$};
    \node[below, my label] at (F) {$F$};
    

    \foreach \point in {A,B,C,D,O,S,M,N,K,T,F} {
      \node[vertex] at (\point) {};
    }
  \end{tikzpicture}
}

\newcommand{\drawPyramidTaskPThree}[2][]{
  \begin{tikzpicture}[my drawing=#2,#1]
    
    \def\x{1.0}
    \def\y{1.0}
    \def\z{1.0}

    \coordinate (A) at (0,0);
    \coordinate (D) at (\x,0);
    \coordinate (C) at (\x * 3 / 8 + \x,\x * 1 / 2);
    \coordinate (B) at (\x * 3 / 8,\x * 1 / 2);

    \coordinate (O) at ($(A)!1/2!(C)$);
    \coordinate (S) at ($(O) + (0, 1.5 * \x)$);

    \coordinate (M) at ($(S)!7/10!(C)$);
    \coordinate (N) at ($(C)!9/10!(D)$);
    

    \draw[dashed] (A) -- (B);
    \draw[dashed] (B) -- (C);
    \draw[dashed] (A) -- (C);
    \draw[dashed] (B) -- (D);
    \draw[dashed] (B) -- (S);
    \draw[dashed] (S) -- (O);
    
    

    \draw[thick] (A) -- (D) -- (C);
    \draw[thick] (A) -- (S);
    \draw[thick] (D) -- (S);
    \draw[thick] (A) -- (S);
    \draw[thick] (C) -- (S);
    
    

    \node[below left, my label] at (A) {$A$};
    \node[below right, my label] at (D) {$D$};
    \node[above right, my label] at (C) {$C$};
    \node[left, my label] at (B) {$B$};
    \node[below, my label] at (O) {$O$};
    \node[above, my label] at (S) {$S$};
    \node[above right, my label] at (M) {$M$};
    \node[above right, my label] at (N) {$N$};

    \foreach \point in {A,B,C,D,O,S,M,N} {
      \node[vertex] at (\point) {};
    }
  \end{tikzpicture}
}


\newcommand{\drawPyramidTaskPFour}[2][]{
  \begin{tikzpicture}[my drawing=#2,#1]
    
    \def\x{1.0}
    \def\y{1.0}
    \def\z{1.0}

    \coordinate (A) at (0,0);
    \coordinate (D) at (\x,0);
    \coordinate (C) at (\x * 3 / 8 + \x,\x * 1 / 2);
    \coordinate (B) at (\x * 3 / 8,\x * 1 / 2);

    \coordinate (O) at ($(A)!1/2!(C)$);
    \coordinate (S) at ($(O) + (0, 1.5 * \x)$);

    \coordinate (M) at ($(S)!7/10!(C)$);
    \coordinate (N) at ($(C)!9/10!(D)$);
    \coordinate (K) at ($(A)!7/10!(C)$);
    \coordinate (T) at ($(B)!11/20!(C)$);
    

    \draw[dashed] (A) -- (B);
    \draw[dashed] (B) -- (C);
    \draw[dashed] (A) -- (C);
    \draw[dashed] (B) -- (D);
    \draw[dashed] (B) -- (S);
    \draw[dashed] (S) -- (O);

    

    \draw[thick] (A) -- (D) -- (C);
    \draw[thick] (A) -- (S);
    \draw[thick] (D) -- (S);
    \draw[thick] (A) -- (S);
    \draw[thick] (C) -- (S);

    \filldraw[fill=blue!15, draw=blue!0, thick, opacity=0.7] 
        (N) -- (M) -- (T) -- cycle;

    \draw[thick, blue] (M) -- (N);
    \draw[dashed, blue] (T) -- (N);
    \draw[dashed, blue] (M) -- (T);
    \draw[dashed, blue] (M) -- (K);
    
    
    

    \node[below left, my label] at (A) {$A$};
    \node[below right, my label] at (D) {$D$};
    \node[above right, my label] at (C) {$C$};
    \node[left, my label] at (B) {$B$};
    \node[below, my label] at (O) {$O$};
    \node[above, my label] at (S) {$S$};
    \node[above right, my label] at (M) {$M$};
    \node[above right, my label] at (N) {$N$};
    \node[above right, my label] at (K) {$K$};
    \node[above, my label] at (T) {$T$};

    \foreach \point in {A,B,C,D,O,S,M,N,K,T} {
      \node[vertex] at (\point) {};
    }
  \end{tikzpicture}
}

\newcommand{\drawPyramidTaskPFive}[2][]{
  \begin{tikzpicture} [my drawing=#2,#1]
    
    \def\x{1.0}
    \def\y{1.0}
    \def\z{1.0}

    \coordinate (A) at (0,0);
    \coordinate (D) at (\x,0);
    \coordinate (C) at (\x * 3 / 8 + \x,\x * 1 / 2);
    \coordinate (B) at (\x * 3 / 8,\x * 1 / 2);

    \coordinate (O) at ($(A)!1/2!(C)$);
    \coordinate (S) at ($(O) + (0, 1.5 * \x)$);

    \coordinate (M) at ($(S)!2/3!(C)$);
    \coordinate (N) at ($(C)!5/10!(D)$);
    \coordinate (K) at ($(A)!2/3!(C)$);

   
    

    \draw[dashed] (A) -- (B);
    \draw[dashed] (B) -- (C);
    \draw[dashed] (A) -- (C);
    \draw[dashed] (B) -- (D);
    \draw[dashed] (B) -- (S);

    
    \draw[thick] (A) -- (D) -- (C);
    \draw[thick] (A) -- (S);
    \draw[thick] (A) -- (S);
    \draw[thick] (C) -- (S);

     \filldraw[fill=blue!15, draw=blue!0, thick, opacity=0.7] 
        (N) -- (M) -- (B) -- cycle;
    
    \draw[thick, blue] (M) -- (N);
    \draw[dashed, blue] (B) -- (N);
    \draw[dashed, blue] (M) -- (B);
    \draw[dashed, blue] (M) -- (K);

    \draw[dashed] (S) -- (O);
    \draw[thick] (D) -- (S);
    
    

    \node[below left, my label] at (A) {$A$};
    \node[below right, my label] at (D) {$D$};
    \node[above right, my label] at (C) {$C$};
    \node[left, my label] at (B) {$B$};
    \node[below, my label] at (O) {$O$};
    \node[above, my label] at (S) {$S$};
    \node[above right, my label] at (M) {$M$};
    \node[above right, my label] at (N) {$N$};
    \node[above right, my label] at (K) {$K$};

    \foreach \point in {A,B,C,D,O,S,M,N,K} {
      \node[vertex] at (\point) {};
    }
  \end{tikzpicture}
}


\newcommand{\drawCubeTaskTwo}[2][]{
  \begin{tikzpicture}[my drawing=#2,#1]
    
    \def\x{1.0}
    \def\y{1.0}
    \def\z{1.0}

    \coordinate (A) at (0,0);
    \coordinate (B) at (\x,0);
    \coordinate (C) at (\x * 4 / 11 * 0.9 + \x,\x * 7 / 11 * 0.3);
    \coordinate (D) at (\x * 4 / 11 * 0.9,\x * 7 / 11 * 0.3);
    \coordinate (A1) at (0,\x);
    \coordinate (B1) at (\x,\x);
    \coordinate (C1) at (\x * 4 / 11 * 0.9 + \x, 1.19);
    \coordinate (D1) at (\x * 4 / 11 * 0.9, 1.19);

    \coordinate (K) at ($(A) !1/2! (D)$);


    \draw[dashed] (A) -- (D);
    \draw[dashed] (B) -- (C);
    \draw[dashed] (D) -- (C);
    \draw[dashed] (D) -- (D1);

    \draw[thick] (A) -- (B) -- (C);
    \draw[thick] (B1) -- (C1);
    \draw[thick] (C) -- (C1);
    \draw[thick] (D1) -- (A1);
    \draw[thick] (C1) -- (D1);
    \draw[thick] (A1) -- (D1);
    \draw[thick] (A) -- (A1);
    

    \filldraw[fill=blue!15, draw=blue!0, thick, opacity=0.7] 
        (K) -- (D1) -- (C) -- cycle;

    \draw[dashed, blue] (K) -- (D1);
    \draw[dashed, blue] (K) -- (C);
    \draw[dashed, blue] (C) -- (D1);


    \draw[thick] (A1) -- (B1);
    \draw[thick] (B) -- (B1);
    

   

    \node[below left, my label] at (A) {$A$};
    \node[above left, my label] at (D) {$D$};
    \node[above right, my label] at (C) {$C$};
    \node[below right, my label] at (B) {$B$};
    \node[below left, my label] at (A1) {$A_1$};
    \node[above left, my label] at (D1) {$D_1$};
    \node[above right, my label] at (C1) {$C_1$};
    \node[below right, my label] at (B1) {$B_1$};
    \node [above left, my label] at (K) {$K$};
    

    \foreach \point in {A,B,C,D,A1,B1,C1,D1,K} {
      \node[vertex] at (\point) {};
    }
  \end{tikzpicture}
}


\newcommand{\drawCubeTaskTwoPTwo}[2][]{
  \begin{tikzpicture}[my drawing=#2,#1]
    
    \def\x{1.0}
    \def\y{1.0}
    \def\z{1.0}

    \coordinate (A) at (0,0);
    \coordinate (B) at (\x,0);
    \coordinate (C) at (\x * 4 / 11 * 0.9 + \x,\x * 7 / 11 * 0.3);
    \coordinate (D) at (\x * 4 / 11 * 0.9,\x * 7 / 11 * 0.3);
    \coordinate (A1) at (0,\x);
    \coordinate (B1) at (\x,\x);
    \coordinate (C1) at (\x * 4 / 11 * 0.9 + \x, 1.19);
    \coordinate (D1) at (\x * 4 / 11 * 0.9, 1.19);

    \draw[dashed] (A) -- (D);
    \draw[dashed] (B) -- (C);
    \draw[dashed] (D) -- (C);
    \draw[dashed] (D) -- (D1);

    \draw[thick] (A) -- (B) -- (C);
    \draw[thick] (B1) -- (C1);
    \draw[thick] (C) -- (C1);
    \draw[thick] (D1) -- (A1);
    \draw[thick] (C1) -- (D1);
    \draw[thick] (A1) -- (D1);
    \draw[thick] (A) -- (A1);
    \draw[thick] (A1) -- (B1);

    \filldraw[fill=blue!15, draw=blue!0, thick, opacity=0.7] 
        (D) -- (B) -- (C1) -- cycle;

    \draw[dashed, blue] (D) -- (B);
    \draw[dashed, blue] (D) -- (C1);
    \draw[thick, blue] (B) -- (C1);

    \draw[thick] (B) -- (B1);

   

    \node[below left, my label] at (A) {$A$};
    \node[above left, my label] at (D) {$D$};
    \node[above right, my label] at (C) {$C$};
    \node[below right, my label] at (B) {$B$};
    \node[below left, my label] at (A1) {$A_1$};
    \node[above left, my label] at (D1) {$D_1$};
    \node[above right, my label] at (C1) {$C_1$};
    \node[below right, my label] at (B1) {$B_1$};
    

    \foreach \point in {A,B,C,D,A1,B1,C1,D1} {
      \node[vertex] at (\point) {};
    }
  \end{tikzpicture}
}


\newcommand{\drawCubeTaskTwoPThree}[2][]{
  \begin{tikzpicture}[my drawing=#2,#1]
    
    \def\x{1.0}
    \def\y{1.0}
    \def\z{1.0}

    \coordinate (A) at (0,0);
    \coordinate (B) at (\x,0);
    \coordinate (C) at (\x * 4 / 11 * 0.9 + \x,\x * 7 / 11 * 0.3);
    \coordinate (D) at (\x * 4 / 11 * 0.9,\x * 7 / 11 * 0.3);
    \coordinate (A1) at (0,\x);
    \coordinate (B1) at (\x,\x);
    \coordinate (C1) at (\x * 4 / 11 * 0.9 + \x, 1.19);
    \coordinate (D1) at (\x * 4 / 11 * 0.9, 1.19);

    \coordinate (A2) at ($(A1) !3/4! (B1)$);
    \coordinate (B2) at ($(B) !4/5! (B1)$);
    \coordinate (C2) at ($(C1) !3/4! (B1)$);
    

    \draw[dashed] (A) -- (D);
    \draw[dashed] (B) -- (C);
    \draw[dashed] (D) -- (C);
    \draw[dashed] (D) -- (D1);

    \draw[thick] (A) -- (B) -- (C);
    \draw[thick] (C) -- (C1);
    \draw[thick] (D1) -- (A1);
    \draw[thick] (C1) -- (D1);
    \draw[thick] (A1) -- (D1);
    \draw[thick] (A) -- (A1);

    \filldraw[fill=blue!15, draw=blue!0, thick, opacity=0.7] 
        (A2) -- (B2) -- (C2) -- cycle;

    \draw[thick, blue] (A2) -- (B2);
    \draw[thick, blue] (A2) -- (C2);
    \draw[thick, blue] (B2) -- (C2);

    \draw[thick] (A1) -- (B1);
    \draw[thick] (B) -- (B1);
    \draw[thick] (B1) -- (C1);
    

   

    \node[below left, my label] at (A) {$A$};
    \node[above left, my label] at (D) {$D$};
    \node[above right, my label] at (C) {$C$};
    \node[below right, my label] at (B) {$B$};
    \node[below left, my label] at (A1) {$A_1$};
    \node[above left, my label] at (D1) {$D_1$};
    \node[above right, my label] at (C1) {$C_1$};
    \node[below right, my label] at (B1) {$B_1$};
    \node[above, my label] at (A2) {$A_2$};
    \node[below right, my label] at (B2) {$B_2$};
    \node[above, my label] at (C2) {$C_2$};
    
    

    \foreach \point in {A,B,C,D,A1,B1,C1,D1,A2,B2,C2} {
      \node[vertex] at (\point) {};
    }
  \end{tikzpicture}
}

\newcommand{\drawCubeTaskTwoPFour}[2][]{
  \begin{tikzpicture}[my drawing=#2,#1]
    
    \def\x{1.0}
    \def\y{1.0}
    \def\z{1.0}

    \coordinate (A) at (0,0);
    \coordinate (B) at (\x,0);
    \coordinate (C) at (\x * 4 / 11 * 0.9 + \x,\x * 7 / 11 * 0.3);
    \coordinate (D) at (\x * 4 / 11 * 0.9,\x * 7 / 11 * 0.3);
    \coordinate (A1) at (0,\x);
    \coordinate (B1) at (\x,\x);
    \coordinate (C1) at (\x * 4 / 11 * 0.9 + \x, 1.19);
    \coordinate (D1) at (\x * 4 / 11 * 0.9, 1.19);

    \coordinate (A2) at ($(A) !6/10! (A1)$);
    \coordinate (B2) at ($(A1) !4/5! (B1)$);
    \coordinate (D2) at ($(A1) !5/8! (D1)$);
    

    \draw[dashed] (A) -- (D);
    \draw[dashed] (B) -- (C);
    \draw[dashed] (D) -- (C);
    \draw[dashed] (D) -- (D1);

    \draw[thick] (A) -- (B) -- (C);
    \draw[thick] (B1) -- (C1);
    \draw[thick] (C) -- (C1);
    \draw[thick] (D1) -- (A1);
    \draw[thick] (C1) -- (D1);
    \draw[thick] (B) -- (B1);
    \draw[thick] (A1) -- (D1);
    \draw[thick] (A) -- (A1);

    \filldraw[fill=blue!15, draw=blue!0, thick, opacity=0.7] 
        (A2) -- (B2) -- (D2) -- cycle;

    \draw[thick, blue] (A2) -- (B2);
    \draw[dashed, blue] (A2) -- (D2);
    \draw[thick, blue] (B2) -- (D2);

    \draw[thick] (A1) -- (B1);

   

    \node[below left, my label] at (A) {$A$};
    \node[above left, my label] at (D) {$D$};
    \node[above right, my label] at (C) {$C$};
    \node[below right, my label] at (B) {$B$};
    \node[below left, my label] at (A1) {$A_1$};
    \node[above, my label] at (D1) {$D_1$};
    \node[above right, my label] at (C1) {$C_1$};
    \node[below right, my label] at (B1) {$B_1$};
    \node[left, my label] at (A2) {$A^\prime$};
    \node[above, my label] at (B2) {$B^\prime$};
    \node[above left, my label] at (D2) {$D^\prime$};
    
    

    \foreach \point in {A,B,C,D,A1,B1,C1,D1,A2,B2,D2} {
      \node[vertex] at (\point) {};
    }
  \end{tikzpicture}
}

\newcommand{\drawCubeTaskTwoPFive}[2][]{
  \begin{tikzpicture}[my drawing=#2,#1]
    
    \def\x{1.0}
    \def\y{1.0}
    \def\z{1.0}

    \coordinate (A) at (0,0);
    \coordinate (B) at (\x,0);
    \coordinate (C) at (\x * 4 / 11 * 0.9 + \x,\x * 7 / 11 * 0.3);
    \coordinate (D) at (\x * 4 / 11 * 0.9,\x * 7 / 11 * 0.3);
    \coordinate (A1) at (0,\x);
    \coordinate (B1) at (\x,\x);
    \coordinate (C1) at (\x * 4 / 11 * 0.9 + \x, 1.19);
    \coordinate (D1) at (\x * 4 / 11 * 0.9, 1.19);

    \coordinate (M) at ($(A) !5/8! (B)$);
    \coordinate (N) at ($(D) !5/8! (C)$);
    \coordinate (E) at ($(A1) !6/8! (B1)$);
    \coordinate (F) at ($(D1) !6/8! (C1)$);
    

    \draw[dashed] (A) -- (D);
    \draw[dashed] (B) -- (C);
    \draw[dashed] (D) -- (C);
    \draw[dashed] (D) -- (D1);

    \draw[thick] (A) -- (B) -- (C);
    \draw[thick] (C) -- (C1);
    \draw[thick] (D1) -- (A1);
    \draw[thick] (C1) -- (D1);
    \draw[thick] (A1) -- (D1);
    \draw[thick] (A) -- (A1);

    \filldraw[fill=blue!15, draw=blue!0, thick, opacity=0.7] 
        (M) -- (N) -- (F) -- (E)-- cycle;

    \draw[thick, blue] (M) -- (E);
    \draw[dashed, blue] (M) -- (N);
    \draw[dashed, blue] (F) -- (N);
    \draw[thick, blue] (E) -- (F);

    \draw[thick] (A1) -- (B1);
    \draw[thick] (B) -- (B1);
    \draw[thick] (B1) -- (C1);
    
    

   

    \node[below left, my label] at (A) {$A$};
    \node[above left, my label] at (D) {$D$};
    \node[above right, my label] at (C) {$C$};
    \node[below right, my label] at (B) {$B$};
    \node[below left, my label] at (A1) {$A_1$};
    \node[above, my label] at (D1) {$D_1$};
    \node[above right, my label] at (C1) {$C_1$};
    \node[below right, my label] at (B1) {$B_1$};

    \node[above, my label] at (F) {$F$};
    \node[above left, my label] at (E) {$E$};
    \node[below, my label] at (M) {$M$};
    \node[above left, my label] at (N) {$N$};
    
    
    

    \foreach \point in {A,B,C,D,A1,B1,C1,D1,N,M,E,F} {
      \node[vertex] at (\point) {};
    }
  \end{tikzpicture}
}

\newcommand{\drawCubeTaskTwoPSix}[2][]{
  \begin{tikzpicture}[my drawing=#2,#1]
    
    \def\x{1.0}
    \def\y{1.0}
    \def\z{1.0}

    \coordinate (A) at (0,0);
    \coordinate (B) at (\x,0);
    \coordinate (C) at (\x * 4 / 11 * 0.9 + \x,\x * 7 / 11 * 0.3);
    \coordinate (D) at (\x * 4 / 11 * 0.9,\x * 7 / 11 * 0.3);
    \coordinate (A1) at (0,\x);
    \coordinate (B1) at (\x,\x);
    \coordinate (C1) at (\x * 4 / 11 * 0.9 + \x, 1.19);
    \coordinate (D1) at (\x * 4 / 11 * 0.9, 1.19);
    

    \draw[dashed] (A) -- (D);
    \draw[dashed] (B) -- (C);
    \draw[dashed] (D) -- (C);
    \draw[dashed] (D) -- (D1);

    \draw[thick] (A) -- (B) -- (C);
    \draw[thick] (C) -- (C1);
    \draw[thick] (D1) -- (A1);
    \draw[thick] (C1) -- (D1);
    \draw[thick] (A1) -- (D1);
    \draw[thick] (A) -- (A1);

    \filldraw[fill=blue!15, draw=blue!0, thick, opacity=0.7] 
        (A) -- (A1) -- (C1) -- (C)-- cycle;

    \draw[thick, blue] (A) -- (A1);
    \draw[dashed, blue] (A) -- (C);
    \draw[dashed, blue] (C) -- (C1);
    \draw[thick, blue] (A1) -- (C1);

    \draw[thick] (A1) -- (B1);
    \draw[thick] (B1) -- (C1);
    \draw[thick] (B) -- (B1);
    

   

    \node[below left, my label] at (A) {$A$};
    \node[above left, my label] at (D) {$D$};
    \node[above right, my label] at (C) {$C$};
    \node[below right, my label] at (B) {$B$};
    \node[below left, my label] at (A1) {$A_1$};
    \node[above, my label] at (D1) {$D_1$};
    \node[above right, my label] at (C1) {$C_1$};
    \node[below right, my label] at (B1) {$B_1$};
    
    
    

    \foreach \point in {A,B,C,D,A1,B1,C1,D1} {
      \node[vertex] at (\point) {};
    }
  \end{tikzpicture}
}

\newcommand{\drawCubeTaskTwoPSeven}[2][]{
  \begin{tikzpicture}[my drawing=#2,#1]
    
    \def\x{1.0}
    \def\y{1.0}
    \def\z{1.0}

    \coordinate (A) at (0,0);
    \coordinate (B) at (\x,0);
    \coordinate (C) at (\x * 4 / 11 * 0.9 + \x,\x * 7 / 11 * 0.3);
    \coordinate (D) at (\x * 4 / 11 * 0.9,\x * 7 / 11 * 0.3);
    \coordinate (A1) at (0,\x);
    \coordinate (B1) at (\x,\x);
    \coordinate (C1) at (\x * 4 / 11 * 0.9 + \x, 1.19);
    \coordinate (D1) at (\x * 4 / 11 * 0.9, 1.19);

    \coordinate (P) at ($(A) !1/2! (B)$);
    \coordinate (K) at ($(D) !1/2! (C)$);
    \coordinate (M1) at ($(A1) !1/2! (B1)$);
    \coordinate (N1) at ($(D1) !1/2! (C1)$);
    

    \draw[dashed] (A) -- (D);
    \draw[dashed] (B) -- (C);
    \draw[dashed] (D) -- (C);
    \draw[dashed] (D) -- (D1);

    \draw[thick] (A) -- (B) -- (C);
    \draw[thick] (B1) -- (C1);
    \draw[thick] (C) -- (C1);
    \draw[thick] (D1) -- (A1);
    \draw[thick] (C1) -- (D1);
    \draw[thick] (B) -- (B1);
    \draw[thick] (A1) -- (D1);
    \draw[thick] (A) -- (A1);

    \filldraw[fill=blue!15, draw=blue!0, thick, opacity=0.7] 
        (P) -- (K) -- (N1) -- (M1)-- cycle;

    \draw[thick, blue] (P) -- (M1);
    \draw[dashed, blue] (P) -- (K);
    \draw[dashed, blue] (N1) -- (K);
    \draw[thick, blue] (M1) -- (N1);

    \draw[thick] (A1) -- (B1);

   

    \node[below left, my label] at (A) {$A$};
    \node[above left, my label] at (D) {$D$};
    \node[above right, my label] at (C) {$C$};
    \node[below right, my label] at (B) {$B$};
    \node[below left, my label] at (A1) {$A_1$};
    \node[above, my label] at (D1) {$D_1$};
    \node[above right, my label] at (C1) {$C_1$};
    \node[below right, my label] at (B1) {$B_1$};

    \node[above, my label] at (N1) {$N$};
    \node[above, my label] at (M1) {$M$};
    \node[below, my label] at (P) {$P$};
    \node[below, my label] at (K) {$K$};
    
    
    \foreach \point in {A,B,C,D,A1,B1,C1,D1,K,P,M1,N1} {
      \node[vertex] at (\point) {};
    }
  \end{tikzpicture}
}

\newcommand{\drawCubeTaskTwoPEight}[2][]{
  \begin{tikzpicture}[my drawing=#2,#1]
    
    \def\x{1.0}
    \def\y{1.0}
    \def\z{1.0}

    \coordinate (A) at (0,0);
    \coordinate (B) at (\x,0);
    \coordinate (C) at (\x * 4 / 11 * 0.9 + \x,\x * 7 / 11 * 0.3);
    \coordinate (D) at (\x * 4 / 11 * 0.9,\x * 7 / 11 * 0.3);
    \coordinate (A1) at (0,\x);
    \coordinate (B1) at (\x,\x);
    \coordinate (C1) at (\x * 4 / 11 * 0.9 + \x, 1.19);
    \coordinate (D1) at (\x * 4 / 11 * 0.9, 1.19);


    \coordinate (M) at ($(D) !1/2! (C)$);
    \coordinate (N) at ($(A1) !1/2! (B1)$);

    

    \draw[dashed] (A) -- (D);
    \draw[dashed] (B) -- (C);
    \draw[dashed] (D) -- (C);
    \draw[dashed] (D) -- (D1);

    \draw[thick] (A) -- (B) -- (C);
    \draw[thick] (C) -- (C1);
    \draw[thick] (D1) -- (A1);
    \draw[thick] (C1) -- (D1);
    \draw[thick] (A1) -- (D1);
    \draw[thick] (A) -- (A1);

    \filldraw[fill=blue!15, draw=blue!0, thick, opacity=0.7] 
        (A) -- (M) -- (C1) -- (N)-- cycle;

    \draw[thick, blue] (A) -- (N);
    \draw[dashed, blue] (A) -- (M);
    \draw[dashed, blue] (C1) -- (M);
    \draw[thick, blue] (N) -- (C1);

    \draw[thick] (A1) -- (B1);
    \draw[thick] (B1) -- (C1);
    \draw[thick] (B) -- (B1);
    

   

    \node[below left, my label] at (A) {$A$};
    \node[above left, my label] at (D) {$D$};
    \node[above right, my label] at (C) {$C$};
    \node[below right, my label] at (B) {$B$};
    \node[below left, my label] at (A1) {$A_1$};
    \node[above, my label] at (D1) {$D_1$};
    \node[above right, my label] at (C1) {$C_1$};
    \node[below right, my label] at (B1) {$B_1$};

    \node[above, my label] at (N) {$N$};
    \node[below, my label] at (M) {$M$};
    
    
    \foreach \point in {A,B,C,D,A1,B1,C1,D1,M,N} {
      \node[vertex] at (\point) {};
    }
  \end{tikzpicture}
}

\newcommand{\drawCubeTaskTwoPNine}[2][]{
  \begin{tikzpicture}[my drawing=#2,#1]
    
    \def\x{1.0}
    \def\y{1.0}
    \def\z{1.0}

    \coordinate (A) at (0,0);
    \coordinate (B) at (\x,0);
    \coordinate (C) at (\x * 4 / 11 * 0.9 + \x,\x * 7 / 11 * 0.3);
    \coordinate (D) at (\x * 4 / 11 * 0.9,\x * 7 / 11 * 0.3);
    \coordinate (A1) at (0,\x);
    \coordinate (B1) at (\x,\x);
    \coordinate (C1) at (\x * 4 / 11 * 0.9 + \x, 1.19);
    \coordinate (D1) at (\x * 4 / 11 * 0.9, 1.19);


    \coordinate (F) at ($(B1) !4/9! (C1)$);
    \coordinate (E) at ($(A1) !4/5! (B1)$);

    

    \draw[dashed] (A) -- (D);
    \draw[dashed] (B) -- (C);
    \draw[dashed] (D) -- (C);
    \draw[dashed] (D) -- (D1);

    \draw[thick] (A) -- (B) -- (C);
    \draw[thick] (C) -- (C1);
    \draw[thick] (D1) -- (A1);
    \draw[thick] (C1) -- (D1);
    \draw[thick] (A1) -- (D1);
    \draw[thick] (A) -- (A1);

    \filldraw[fill=blue!15, draw=blue!0, thick, opacity=0.7] 
        (A) -- (E) -- (F) -- (C)-- cycle;

    \draw[thick, blue] (A) -- (E);
    \draw[dashed, blue] (A) -- (C);
    \draw[thick, blue] (C) -- (F);
    \draw[thick, blue] (E) -- (F);

    \draw[thick] (A1) -- (B1);
    \draw[thick] (B) -- (B1);
    \draw[thick] (B1) -- (C1);
    

   

    \node[below left, my label] at (A) {$A$};
    \node[above left, my label] at (D) {$D$};
    \node[above right, my label] at (C) {$C$};
    \node[below right, my label] at (B) {$B$};
    \node[below left, my label] at (A1) {$A_1$};
    \node[above, my label] at (D1) {$D_1$};
    \node[above right, my label] at (C1) {$C_1$};
    \node[below right, my label] at (B1) {$B_1$};

    \node[right, my label] at (F) {$F$};
    \node[above, my label] at (E) {$E$};
    
    
    \foreach \point in {A,B,C,D,A1,B1,C1,D1,E,F} {
      \node[vertex] at (\point) {};
    }
  \end{tikzpicture}
}

\newcommand{\drawCubeTaskTwoPTen}[2][]{
  \begin{tikzpicture}[my drawing=#2,#1]
    
    \def\x{1.0}
    \def\y{1.0}
    \def\z{1.0}

    \coordinate (A) at (0,0);
    \coordinate (B) at (\x,0);
    \coordinate (C) at (\x * 4 / 11 * 0.9 + \x,\x * 7 / 11 * 0.3);
    \coordinate (D) at (\x * 4 / 11 * 0.9,\x * 7 / 11 * 0.3);
    \coordinate (A1) at (0,\x);
    \coordinate (B1) at (\x,\x);
    \coordinate (C1) at (\x * 4 / 11 * 0.9 + \x, 1.19);
    \coordinate (D1) at (\x * 4 / 11 * 0.9, 1.19);


    \coordinate (N) at ($(B1) !1/2! (C1)$);
    \coordinate (M) at ($(A1) !7/10! (B1)$);

    

    \draw[dashed] (A) -- (D);
    \draw[dashed] (B) -- (C);
    \draw[dashed] (D) -- (C);
    \draw[dashed] (D) -- (D1);

    \draw[thick] (A) -- (B) -- (C);
    \draw[thick] (C) -- (C1);
    \draw[thick] (D1) -- (A1);
    \draw[thick] (C1) -- (D1);
    \draw[thick] (A1) -- (D1);
    \draw[thick] (A) -- (A1);

    \filldraw[fill=blue!15, draw=blue!0, thick, opacity=0.7] 
        (A) -- (M) -- (N) -- (C)-- cycle;

    \draw[thick, blue] (A) -- (M);
    \draw[dashed, blue] (A) -- (C);
    \draw[thick, blue] (C) -- (N);
    \draw[thick, blue] (M) -- (N);

    \draw[thick] (A1) -- (B1);
    \draw[thick] (B) -- (B1);
    \draw[thick] (B1) -- (C1);
    

   

    \node[below left, my label] at (A) {$A$};
    \node[above left, my label] at (D) {$D$};
    \node[above right, my label] at (C) {$C$};
    \node[below right, my label] at (B) {$B$};
    \node[below left, my label] at (A1) {$A_1$};
    \node[above, my label] at (D1) {$D_1$};
    \node[above right, my label] at (C1) {$C_1$};
    \node[below right, my label] at (B1) {$B_1$};

    \node[right, my label] at (N) {$N$};
    \node[above, my label] at (M) {$M$};
    
    
    \foreach \point in {A,B,C,D,A1,B1,C1,D1,M,N} {
      \node[vertex] at (\point) {};
    }
  \end{tikzpicture}
}


\newcommand{\drawCubeTaskTwoPEleven}[2][]{
  \begin{tikzpicture}[my drawing=#2,#1]
    
    \def\x{1.0}
    \def\y{1.0}
    \def\z{1.0}

    \coordinate (A) at (0,0);
    \coordinate (B) at (\x,0);
    \coordinate (C) at (\x * 4 / 11 * 0.9 + \x,\x * 7 / 11 * 0.3);
    \coordinate (D) at (\x * 4 / 11 * 0.9,\x * 7 / 11 * 0.3);
    \coordinate (A1) at (0,\x);
    \coordinate (B1) at (\x,\x);
    \coordinate (C1) at (\x * 4 / 11 * 0.9 + \x, 1.19);
    \coordinate (D1) at (\x * 4 / 11 * 0.9, 1.19);


    \coordinate (C2) at ($(B1) !1/2! (C1)$);
    \coordinate (B2) at ($(B) !1/2! (C)$);

    

    \draw[dashed] (A) -- (D);
    \draw[dashed] (B) -- (C);
    \draw[dashed] (D) -- (C);
    \draw[dashed] (D) -- (D1);

    \draw[thick] (A) -- (B) -- (C);
    \draw[thick] (B1) -- (C1);
    \draw[thick] (C) -- (C1);
    \draw[thick] (D1) -- (A1);
    \draw[thick] (C1) -- (D1);
    \draw[thick] (B) -- (B1);
    \draw[thick] (A1) -- (D1);
    \draw[thick] (A) -- (A1);
    \draw[thick] (A1) -- (B1);


    \draw[thick, blue] (B2) -- (C2);

   

    \node[below left, my label] at (A) {$A$};
    \node[above left, my label] at (D) {$D$};
    \node[above right, my label] at (C) {$C$};
    \node[below right, my label] at (B) {$B$};
    \node[below left, my label] at (A1) {$A_1$};
    \node[above, my label] at (D1) {$D_1$};
    \node[above right, my label] at (C1) {$C_1$};
    \node[below right, my label] at (B1) {$B_1$};

    \node[below right, my label] at (B2) {$B_2$};
    \node[above left, my label] at (C2) {$C_2$};
    
    
    \foreach \point in {A,B,C,D,A1,B1,C1,D1,B2,C2} {
      \node[vertex] at (\point) {};
    }
  \end{tikzpicture}
}

\newcommand{\drawCubeTaskTwoPTwelve}[2][]{
  \begin{tikzpicture}[my drawing=#2,#1]
    
    \def\x{1.0}
    \def\y{1.0}
    \def\z{1.0}

    \coordinate (A) at (0,0);
    \coordinate (B) at (\x,0);
    \coordinate (C) at (\x * 4 / 11 * 0.9 + \x,\x * 7 / 11 * 0.3);
    \coordinate (D) at (\x * 4 / 11 * 0.9,\x * 7 / 11 * 0.3);
    \coordinate (A1) at (0,\x);
    \coordinate (B1) at (\x,\x);
    \coordinate (C1) at (\x * 4 / 11 * 0.9 + \x, 1.19);
    \coordinate (D1) at (\x * 4 / 11 * 0.9, 1.19);


    \coordinate (C2) at ($(B1) !1/2! (C1)$);
    \coordinate (B2) at ($(B) !1/2! (C)$);
    \coordinate (D2) at ($(A1) !2/3! (B1)$);
    \coordinate (A2) at ($(A) !5/12! (B)$);

    

    \draw[dashed] (A) -- (D);
    \draw[dashed] (B) -- (C);
    \draw[dashed] (D) -- (C);
    \draw[dashed] (D) -- (D1);

    \draw[thick] (A) -- (B) -- (C);
    \draw[thick] (C) -- (C1);
    \draw[thick] (D1) -- (A1);
    \draw[thick] (C1) -- (D1);
    \draw[thick] (A1) -- (D1);
    \draw[thick] (A) -- (A1);

    \filldraw[fill=blue!15, draw=blue!0, thick, opacity=0.7] 
            (A2) -- (D2) -- (C2) -- (B2)-- cycle;

    \draw[thick, blue] (B2) -- (C2);
    \draw[dashed, blue] (A2) -- (B2);
    \draw[thick, blue] (D2) -- (C2);
    \draw[thick, blue] (D2) -- (A2);


    \draw[thick] (A1) -- (B1);
    \draw[thick] (B1) -- (C1);
    \draw[thick] (B) -- (B1);

   

    \node[below left, my label] at (A) {$A$};
    \node[above left, my label] at (D) {$D$};
    \node[above right, my label] at (C) {$C$};
    \node[below right, my label] at (B) {$B$};
    \node[below left, my label] at (A1) {$A_1$};
    \node[above, my label] at (D1) {$D_1$};
    \node[above right, my label] at (C1) {$C_1$};
    \node[below right, my label] at (B1) {$B_1$};

    \node[below right, my label] at (B2) {$B_2$};
    \node[above left, my label] at (C2) {$C_2$};
    \node[above left, my label] at (D2) {$D_2$};
    \node[below, my label] at (A2) {$A_2$};
    
    
    \foreach \point in {A,B,C,D,A1,B1,C1,D1,B2,C2,A2,D2} {
      \node[vertex] at (\point) {};
    }
  \end{tikzpicture}
}

\newcommand{\drawCubeTaskTwoPThirteen}[2][]{
  \begin{tikzpicture}[my drawing=#2,#1]
    
    \def\x{1.0}
    \def\y{1.0}
    \def\z{1.0}

    \coordinate (A) at (0,0);
    \coordinate (B) at (\x,0);
    \coordinate (C) at (\x * 4 / 11 * 0.9 + \x,\x * 7 / 11 * 0.3);
    \coordinate (D) at (\x * 4 / 11 * 0.9,\x * 7 / 11 * 0.3);
    \coordinate (A1) at (0,\x);
    \coordinate (B1) at (\x,\x);
    \coordinate (C1) at (\x * 4 / 11 * 0.9 + \x, 1.19);
    \coordinate (D1) at (\x * 4 / 11 * 0.9, 1.19);


    \coordinate (C2) at ($(B1) !1/2! (C1)$);
    \coordinate (B2) at ($(B) !1/2! (C)$);
    \coordinate (D2) at ($(A1) !2/3! (D1)$);
    \coordinate (A2) at ($(A) !2/3! (D)$);

    

    \draw[dashed] (A) -- (D);
    \draw[dashed] (B) -- (C);
    \draw[dashed] (D) -- (C);
    \draw[dashed] (D) -- (D1);

    \draw[thick] (A) -- (B) -- (C);
    \draw[thick] (C) -- (C1);
    \draw[thick] (D1) -- (A1);
    \draw[thick] (C1) -- (D1);
    \draw[thick] (A1) -- (D1);
    \draw[thick] (A) -- (A1);

    \filldraw[fill=blue!15, draw=blue!0, thick, opacity=0.7] 
            (A2) -- (D2) -- (C2) -- (B2)-- cycle;

    \draw[thick, blue] (B2) -- (C2);
    \draw[dashed, blue] (A2) -- (B2);
    \draw[thick, blue] (D2) -- (C2);
    \draw[dashed, blue] (D2) -- (A2);

    \draw[thick] (A1) -- (B1);
    \draw[thick] (B) -- (B1);
    \draw[thick] (B1) -- (C1);
    

   

    \node[below left, my label] at (A) {$A$};
    \node[above left, my label] at (D) {$D$};
    \node[above right, my label] at (C) {$C$};
    \node[below right, my label] at (B) {$B$};
    \node[below left, my label] at (A1) {$A_1$};
    \node[above, my label] at (D1) {$D_1$};
    \node[above right, my label] at (C1) {$C_1$};
    \node[below right, my label] at (B1) {$B_1$};

    \node[below right, my label] at (B2) {$B_2$};
    \node[above left, my label] at (C2) {$C_2$};
    \node[above left, my label] at (D2) {$D_2$};
    \node[below, my label] at (A2) {$A_2$};
    
    
    \foreach \point in {A,B,C,D,A1,B1,C1,D1,B2,C2,A2,D2} {
      \node[vertex] at (\point) {};
    }
  \end{tikzpicture}
}

\newcommand{\drawCubeTaskTwoPFourteen}[2][]{
  \begin{tikzpicture}[my drawing=#2,#1]
    
    \def\x{1.0}
    \def\y{1.0}
    \def\z{1.0}

    \coordinate (A) at (0,0);
    \coordinate (B) at (\x,0);
    \coordinate (C) at (\x * 4 / 11 * 0.9 + \x,\x * 7 / 11 * 0.3);
    \coordinate (D) at (\x * 4 / 11 * 0.9,\x * 7 / 11 * 0.3);
    \coordinate (A1) at (0,\x);
    \coordinate (B1) at (\x,\x);
    \coordinate (C1) at (\x * 4 / 11 * 0.9 + \x, 1.19);
    \coordinate (D1) at (\x * 4 / 11 * 0.9, 1.19);


    \coordinate (K) at ($(D) !1/2! (D1)$);
    \coordinate (M) at ($(A) !1/3! (A1)$);
    \coordinate (N) at ($(C) !1/3! (C1)$);
    
    

    

    \draw[dashed] (A) -- (D);
    \draw[dashed] (B) -- (C);
    \draw[dashed] (D) -- (C);
    \draw[dashed] (D) -- (D1);

    \draw[thick] (A) -- (B) -- (C);
    \draw[thick] (C) -- (C1);
    \draw[thick] (D1) -- (A1);
    \draw[thick] (C1) -- (D1);
    \draw[thick] (A1) -- (D1);
    \draw[thick] (A) -- (A1);



    \draw[dashed, blue] (K) -- (M);
    \draw[dashed, blue] (K) -- (N);
    
    
    \draw[thick] (A1) -- (B1);
    \draw[thick] (B) -- (B1);
    \draw[thick] (B1) -- (C1);

   

    \node[below left, my label] at (A) {$A$};
    \node[above left, my label] at (D) {$D$};
    \node[above right, my label] at (C) {$C$};
    \node[below right, my label] at (B) {$B$};
    \node[below left, my label] at (A1) {$A_1$};
    \node[above, my label] at (D1) {$D_1$};
    \node[above right, my label] at (C1) {$C_1$};
    \node[below right, my label] at (B1) {$B_1$};

    \node[above left, my label] at (K) {$K$};
    \node[left, my label] at (M) {$M$};

    \node[right, my label] at (N) {$N$};
    
    
    
    
    \foreach \point in {A,B,C,D,A1,B1,C1,D1,M,K,N} {
      \node[vertex] at (\point) {};
    }
  \end{tikzpicture}
}


\newcommand{\drawCubeTaskTwoPFifteen}[2][]{
  \begin{tikzpicture}[my drawing=#2,#1]
    
    \def\x{1.0}
    \def\y{1.0}
    \def\z{1.0}

    \coordinate (A) at (0,0);
    \coordinate (B) at (\x,0);
    \coordinate (C) at (\x * 4 / 11 * 0.9 + \x,\x * 7 / 11 * 0.3);
    \coordinate (D) at (\x * 4 / 11 * 0.9,\x * 7 / 11 * 0.3);
    \coordinate (A1) at (0,\x);
    \coordinate (B1) at (\x,\x);
    \coordinate (C1) at (\x * 4 / 11 * 0.9 + \x, 1.19);
    \coordinate (D1) at (\x * 4 / 11 * 0.9, 1.19);


    \coordinate (K) at ($(D) !1/2! (D1)$);
    \coordinate (M) at ($(A) !1/2! (A1)$);
    \coordinate (P) at ($(B) !1/3! (B1)$);
    \coordinate (N) at ($(C) !1/3! (C1)$);
    
    

    

    \draw[dashed] (A) -- (D);
    \draw[dashed] (B) -- (C);
    \draw[dashed] (D) -- (C);
    \draw[dashed] (D) -- (D1);

    \draw[thick] (A) -- (B) -- (C);
    \draw[thick] (C) -- (C1);
    \draw[thick] (D1) -- (A1);
    \draw[thick] (C1) -- (D1);
    \draw[thick] (A1) -- (D1);
    \draw[thick] (A) -- (A1);

    \filldraw[fill=blue!15, draw=blue!0, thick, opacity=0.7] 
            (K) -- (M) -- (P) -- (N) -- cycle;


    \draw[dashed, blue] (K) -- (M);
    \draw[dashed, blue] (K) -- (N);
    \draw[thick, blue] (M) -- (P);
    \draw[thick, blue] (P) -- (N);
    
    
    \draw[thick] (A1) -- (B1);
    \draw[thick] (B) -- (B1);
    \draw[thick] (B1) -- (C1);

   

    \node[below left, my label] at (A) {$A$};
    \node[above left, my label] at (D) {$D$};
    \node[above right, my label] at (C) {$C$};
    \node[below right, my label] at (B) {$B$};
    \node[below left, my label] at (A1) {$A_1$};
    \node[above, my label] at (D1) {$D_1$};
    \node[above right, my label] at (C1) {$C_1$};
    \node[below right, my label] at (B1) {$B_1$};

    \node[above left, my label] at (K) {$K$};
    \node[left, my label] at (M) {$M$};
    \node[below right, my label] at (P) {$P$};
    \node[right, my label] at (N) {$N$};
    
    
    
    
    \foreach \point in {A,B,C,D,A1,B1,C1,D1,M,K,N,P} {
      \node[vertex] at (\point) {};
    }
  \end{tikzpicture}
}


\newcommand{\drawCubeTaskTwoPSixteen}[2][]{
  \begin{tikzpicture}[my drawing=#2,#1]
    
    \def\x{1.0}
    \def\y{1.0}
    \def\z{1.0}

    \coordinate (A) at (0,0);
    \coordinate (B) at (\x,0);
    \coordinate (C) at (\x * 4 / 11 * 0.9 + \x,\x * 7 / 11 * 0.3);
    \coordinate (D) at (\x * 4 / 11 * 0.9,\x * 7 / 11 * 0.3);
    \coordinate (A1) at (0,\x);
    \coordinate (B1) at (\x,\x);
    \coordinate (C1) at (\x * 4 / 11 * 0.9 + \x, 1.19);
    \coordinate (D1) at (\x * 4 / 11 * 0.9, 1.19);


    \coordinate (N) at ($(D) !2/3! (D1)$);
    \coordinate (M) at ($(A) !1/3! (A1)$);
    \coordinate (T) at ($(A) !2/3! (B)$);
    \coordinate (K) at ($(B) !1/2! (C)$);
    \coordinate (P) at ($(C) !1/6! (C1)$);
    
    

    

    \draw[dashed] (A) -- (D);
    \draw[dashed] (B) -- (C);
    \draw[dashed] (D) -- (C);
    \draw[dashed] (D) -- (D1);

    \draw[thick] (A) -- (B) -- (C);
    \draw[thick] (C) -- (C1);
    \draw[thick] (D1) -- (A1);
    \draw[thick] (C1) -- (D1);
    \draw[thick] (A1) -- (D1);
    \draw[thick] (A) -- (A1);

    \filldraw[fill=blue!15, draw=blue!0, thick, opacity=0.7] 
            (N) -- (M) -- (T) -- (K) -- (P) -- cycle;


    \draw[dashed, blue] (N) -- (M);
    \draw[dashed, blue] (N) -- (P);
    \draw[dashed, blue] (T) -- (K);
    \draw[thick, blue] (K) -- (P);
    \draw[thick, blue] (T) -- (M);
    
    
    \draw[thick] (A1) -- (B1);
    \draw[thick] (B) -- (B1);
    \draw[thick] (B1) -- (C1);

   

   \node[below left, my label] at (A) {$A$};
    \node[above left, my label] at (D) {$D$};
    \node[below right, my label] at (C) {$C$};
    \node[below right, my label] at (B) {$B$};
    \node[above left, my label] at (A1) {$A_1$};
    \node[above, my label] at (D1) {$D_1$};
    \node[above right, my label] at (C1) {$C_1$};
    \node[below right, my label] at (B1) {$B_1$};

    \node[below right, my label] at (K) {$P$};
    \node[left, my label] at (M) {$M$};
    \node[right, my label] at (P) {$N$};
    \node[below, my label] at (T) {$T$};
    \node[above left, my label] at (N) {$K$};
    
    
    
    
    \foreach \point in {A,B,C,D,A1,B1,C1,D1,M,K,N,T,P} {
      \node[vertex] at (\point) {};
    }
  \end{tikzpicture}
}


\newcommand{\drawCubeTaskTwoPSeventeen}[2][]{
  \begin{tikzpicture}[my drawing=#2,#1]
    
    \def\x{1.0}
    \def\y{1.0}
    \def\z{1.0}

    \coordinate (A) at (0,0);
    \coordinate (B) at (\x,0);
    \coordinate (C) at (\x * 4 / 11 * 0.9 + \x,\x * 7 / 11 * 0.3);
    \coordinate (D) at (\x * 4 / 11 * 0.9,\x * 7 / 11 * 0.3);
    \coordinate (A1) at (0,\x);
    \coordinate (B1) at (\x,\x);
    \coordinate (C1) at (\x * 4 / 11 * 0.9 + \x, 1.19);
    \coordinate (D1) at (\x * 4 / 11 * 0.9, 1.19);


    \coordinate (P) at ($(D1) !1/2! (C1)$);
    \coordinate (K) at ($(D) !1/2! (C)$);
    \coordinate (N) at ($(A1) !1/2! (D1)$);
    \coordinate (M) at ($(A) !1/2! (D)$);

    

    \draw[dashed] (A) -- (D);
    \draw[dashed] (B) -- (C);
    \draw[dashed] (D) -- (C);
    \draw[dashed] (D) -- (D1);

    \draw[thick] (A) -- (B) -- (C);
    \draw[thick] (C) -- (C1);
    \draw[thick] (D1) -- (A1);
    \draw[thick] (C1) -- (D1);
    \draw[thick] (A1) -- (D1);
    \draw[thick] (A) -- (A1);

    \filldraw[fill=blue!15, draw=blue!0, thick, opacity=0.7] 
            (M) -- (N) -- (P) -- (K)-- cycle;

    \draw[dashed, blue] (K) -- (P);
    \draw[dashed, blue] (M) -- (K);
    \draw[thick, blue] (N) -- (P);
    \draw[dashed, blue] (N) -- (M);


    \draw[thick] (A1) -- (B1);
    \draw[thick] (B1) -- (C1);
    \draw[thick] (B) -- (B1);

   

    \node[below left, my label] at (A) {$A$};
    \node[above left, my label] at (D) {$D$};
    \node[above right, my label] at (C) {$C$};
    \node[below right, my label] at (B) {$B$};
    \node[below left, my label] at (A1) {$A_1$};
    \node[above, my label] at (D1) {$D_1$};
    \node[above right, my label] at (C1) {$C_1$};
    \node[below right, my label] at (B1) {$B_1$};

    \node[below right, my label] at (K) {$K$};
    \node[above left, my label] at (P) {$P$};
    \node[above left, my label] at (N) {$N$};
    \node[below, my label] at (M) {$M$};
    
    
    \foreach \point in {A,B,C,D,A1,B1,C1,D1,K,P,M,N} {
      \node[vertex] at (\point) {};
    }
  \end{tikzpicture}
}

\newcommand{\drawCubeTaskTwoPEighteen}[2][]{
  \begin{tikzpicture}[my drawing=#2,#1]
    
    \def\x{1.0}
    \def\y{1.0}
    \def\z{1.0}

    \coordinate (A) at (0,0);
    \coordinate (B) at (\x,0);
    \coordinate (C) at (\x * 4 / 11 * 0.9 + \x,\x * 7 / 11 * 0.3);
    \coordinate (D) at (\x * 4 / 11 * 0.9,\x * 7 / 11 * 0.3);
    \coordinate (A1) at (0,\x);
    \coordinate (B1) at (\x,\x);
    \coordinate (C1) at (\x * 4 / 11 * 0.9 + \x, 1.19);
    \coordinate (D1) at (\x * 4 / 11 * 0.9, 1.19);


    \coordinate (N) at ($(A1) !49/96! (D1)$);
    \coordinate (M) at ($(A) !12/15! (A1)$);
    \coordinate (T) at ($(A) !2/3! (B)$);
    \coordinate (K) at ($(B) !1/2! (C)$);
    \coordinate (P) at ($(C) !1/6! (C1)$);
    
    \coordinate (F) at ($(N) + (K) - (T)$);

    

    \draw[dashed] (A) -- (D);
    \draw[dashed] (B) -- (C);
    \draw[dashed] (D) -- (C);
    \draw[dashed] (D) -- (D1);

    \draw[thick] (A) -- (B) -- (C);
    \draw[thick] (C) -- (C1);
    \draw[thick] (D1) -- (A1);
    \draw[thick] (C1) -- (D1);
    \draw[thick] (A1) -- (D1);
    \draw[thick] (A) -- (A1);

    \filldraw[fill=blue!15, draw=blue!0, thick, opacity=0.7] 
            (F) -- (N) -- (M) -- (T) -- (K) -- (P) -- cycle;


    \draw[dashed, blue] (N) -- (M);
    \draw[thick, blue] (N) -- (F);
    \draw[dashed, blue] (F) -- (P);
    \draw[dashed, blue] (T) -- (K);
    \draw[thick, blue] (K) -- (P);
    \draw[thick, blue] (T) -- (M);
    
    
    \draw[thick] (A1) -- (B1);
    \draw[thick] (B) -- (B1);
    \draw[thick] (B1) -- (C1);

   

    \node[below left, my label] at (A) {$A$};
    \node[above left, my label] at (D) {$D$};
    \node[below right, my label] at (C) {$C$};
    \node[below right, my label] at (B) {$B$};
    \node[above left, my label] at (A1) {$A_1$};
    \node[above, my label] at (D1) {$D_1$};
    \node[above right, my label] at (C1) {$C_1$};
    \node[below right, my label] at (B1) {$B_1$};

    \node[below right, my label] at (K) {$P$};
    \node[left, my label] at (M) {$M$};
    \node[right, my label] at (P) {$N$};
    \node[below, my label] at (T) {$T$};
    \node[above left, my label] at (N) {$K$};
    \node[above left, my label] at (F) {$F$};
    
    
    
    
    \foreach \point in {A,B,C,D,A1,B1,C1,D1,M,K,N,T,P,F} {
      \node[vertex] at (\point) {};
    }
  \end{tikzpicture}
}

\newcommand{\drawTetrahedron}[2][]{
  \begin{tikzpicture}[my drawing=#2,#1]
    
    \def\x{1.0 / 6}
    \def\y{1.0}
    \def\z{1.0}

    \coordinate (A) at (0,0);
    \coordinate (B) at (\x * 6, \x * 3);
    \coordinate (C) at (\x * 9, -\x);
    \coordinate (A1) at ($(A) + (0, 1.41 * \x * 3)$);
    \coordinate (B1) at ($(B) + (0, 1.41 * \x * 3)$);
    \coordinate (C1) at ($(C) + (0, 1.41 * \x * 3)$);


    \coordinate (K) at ($(B1) !2/3! (C1)$);
    \coordinate (N) at ($(A1) !2/3! (B1)$);
    \coordinate (M) at ($(A) !1/3! (B)$);



    

    \draw[dashed] (B) -- (C);
    \draw[dashed] (A) -- (B);
    \draw[dashed] (B) -- (B1);

    \draw[thick] (A) -- (C);
    \draw[thick] (C) -- (C1);
    \draw[thick] (A) -- (A1);
    \draw[thick] (B1) -- (A1);
    \draw[thick] (C1) -- (A1);
    \draw[thick] (C1) -- (B1);

    \draw[thick, blue] (K) -- (N);
    \draw[dashed, blue] (M) -- (N);
    


   

    \node[below left, my label] at (A) {$A$};
    \node[below right, my label] at (C) {$C$};
    \node[right, my label] at (B) {$B$};
    \node[left, my label] at (A1) {$A_1$};
    \node[right, my label] at (C1) {$C_1$};
    \node[above, my label] at (B1) {$B_1$};

    \node[right, my label] at (K) {$K$};
    \node[above left, my label] at (N) {$N$};
    \node[above left, my label] at (M) {$M$};

    
    \path (A1) -- (A) node[midway, left, my formula] {$\sqrt{2}$};
    \path (A) -- (M) node[midway, above, sloped, my formula] {$1$};
    \path (M) -- (B) node[midway, above, sloped, my formula] {$2$};
    \path (A1) -- (N) node[midway, above, sloped, my formula] {$2$};
    \path (N) -- (B1) node[midway, above, sloped, my formula] {$1$};
    \path (B1) -- (K) node[midway, above, sloped, my formula] {$2$};
    \path (K) -- (C1) node[midway, above, sloped, my formula] {$1$};
    
    

     
    \foreach \point in {A,B,C,A1,B1,C1,N,K,M} {
      \node[vertex] at (\point) {};
    }
  \end{tikzpicture}
}

\newcommand{\drawTetrahedronPTwo}[2][]{
  \begin{tikzpicture}[my drawing=#2,#1]
    
    \def\x{1.0 / 6}
    \def\y{1.0}
    \def\z{1.0}

    \coordinate (A) at (0,0);
    \coordinate (B) at (\x * 6, \x * 3);
    \coordinate (C) at (\x * 9, -\x);
    \coordinate (A1) at ($(A) + (0, 1.41 * \x * 3)$);
    \coordinate (B1) at ($(B) + (0, 1.41 * \x * 3)$);
    \coordinate (C1) at ($(C) + (0, 1.41 * \x * 3)$);


    \coordinate (K) at ($(B1) !2/3! (C1)$);
    \coordinate (N) at ($(A1) !2/3! (B1)$);
    \coordinate (M) at ($(A) !1/3! (B)$);
    \coordinate (P) at ($(B) !7/9! (C)$);

    

    \draw[dashed] (B) -- (C);
    \draw[dashed] (A) -- (B);
    \draw[dashed] (B) -- (B1);

    \draw[thick] (A) -- (C);
    \draw[thick] (C) -- (C1);
    \draw[thick] (A) -- (A1);
    \draw[thick] (B1) -- (A1);
    \draw[thick] (C1) -- (B1);

    \filldraw[fill=blue!15, draw=blue!0, thick, opacity=0.7] 
            (M) -- (N) -- (K) -- (P)-- cycle;

    \draw[thick, blue] (K) -- (N);
    \draw[dashed, blue] (M) -- (N);
    \draw[dashed, blue] (M) -- (P);
    \draw[dashed, blue] (K) -- (P);
    

    \draw[thick] (C1) -- (A1);
   

    \node[below left, my label] at (A) {$A$};
    \node[below right, my label] at (C) {$C$};
    \node[right, my label] at (B) {$B$};
    \node[left, my label] at (A1) {$A_1$};
    \node[right, my label] at (C1) {$C_1$};
    \node[above, my label] at (B1) {$B_1$};

    \node[right, my label] at (K) {$K$};
    \node[above left, my label] at (N) {$N$};
    \node[above left, my label] at (M) {$M$};
    \node[right, my label] at (P) {$P$};

   
    \path (A1) -- (A) node[midway, left, my formula] {$\sqrt{2}$};
    \path (A) -- (M) node[midway, above, sloped, my formula] {$1$};
    \path (M) -- (B) node[midway, above, sloped, my formula] {$2$};
    \path (A1) -- (N) node[midway, above, sloped, my formula] {$2$};
    \path (N) -- (B1) node[midway, above, sloped, my formula] {$1$};
    \path (B1) -- (K) node[midway, above, sloped, my formula] {$2$};
    \path (K) -- (C1) node[midway, above, sloped, my formula] {$1$};
    
    
    \foreach \point in {A,B,C,A1,B1,C1,N,K,M,P} {
      \node[vertex] at (\point) {};
    }
  \end{tikzpicture}
}

\newcommand{\drawTetrahedronPThree}[2][]{
  \begin{tikzpicture}[my drawing=#2,#1]
    
    \def\x{1.0 / 6}
    \def\y{1.0}
    \def\z{1.0}

    \coordinate (A) at (0,0);
    \coordinate (B) at (\x * 6, \x * 3);
    \coordinate (C) at (\x * 9, -\x);
    \coordinate (A1) at ($(A) + (0, 1.41 * \x * 3)$);
    \coordinate (B1) at ($(B) + (0, 1.41 * \x * 3)$);
    \coordinate (C1) at ($(C) + (0, 1.41 * \x * 3)$);


    \coordinate (K) at ($(B1) !2/3! (C1)$);
    \coordinate (N) at ($(A1) !2/3! (B1)$);
    \coordinate (M) at ($(A) !1/3! (B)$);
    \coordinate (F) at ($(B) !1/3! (A)$);
    \coordinate (S) at ($(B) !2/3! (C)$);
    \coordinate (H) at ($(A1) !1/3! (B1)$);
    
    \coordinate (L) at ($(M) + (K) - (N)$);
    
    

    

    \draw[dashed] (B) -- (C);
    \draw[dashed] (A) -- (B);
    \draw[dashed] (B) -- (B1);
    \draw[dashed] (K) -- (S);
    \draw[dashed] (F) -- (N);
    \draw[dashed] (M) -- (H);
    \draw[dashed] (F) -- (L);
    \draw[dashed] (M) -- (S);
    \draw[dashed] (K) -- (L);
    
    

    \draw[thick] (A) -- (C);
    \draw[thick] (C) -- (C1);
    \draw[thick] (A) -- (A1);
    \draw[thick] (B1) -- (A1);
    \draw[thick] (C1) -- (B1);


    \draw[thick, blue] (K) -- (N);
    \draw[dashed, blue] (M) -- (N);
    \draw[dashed, blue] (M) -- (L);
    
    

    \draw[thick] (C1) -- (A1);
   

    \node[below left, my label] at (A) {$A$};
    \node[below right, my label] at (C) {$C$};
    \node[right, my label] at (B) {$B$};
    \node[left, my label] at (A1) {$A_1$};
    \node[right, my label] at (C1) {$C_1$};
    \node[above, my label] at (B1) {$B_1$};

    \node[right, my label] at (K) {$K$};
    \node[above left, my label] at (N) {$N$};
    \node[above left, my label] at (M) {$M$};
    \node[above left, my label] at (F) {$F$};
    \node[right, my label] at (S) {$S$};
    \node[above, my label] at (H) {$H$};
    \node[below, my label] at (L) {$L$};
    
    \pic [draw, thick, angle radius=2mm, angle eccentricity=1.2] {right angle = M--H--N};
    \pic [draw, thick, angle radius=2mm, angle eccentricity=1.2] {right angle = N--F--B};
    \pic [draw, thick, angle radius=2mm, angle eccentricity=1.2] {right angle = K--S--B};
    
    \path (A1) -- (A) node[midway, left, my formula] {$\sqrt{2}$};
    \path (A) -- (M) node[midway, above, sloped, my formula] {$1$};
    \path (F) -- (B) node[midway, above, sloped, my formula] {$1$};
    \path (H) -- (N) node[midway, above, sloped, my formula] {$1$};
    \path (B1) -- (K) node[midway, above, sloped, my formula] {$2$};
    \path (K) -- (C1) node[midway, above, sloped, my formula] {$1$};
    
    \foreach \point in {A,B,C,A1,B1,C1,N,K,M,F,S,L,H} {
      \node[vertex] at (\point) {};
    }
  \end{tikzpicture}
}


\newcommand{\drawTetrahedronPFour}[2][]{
  \begin{tikzpicture}[my drawing=#2,#1]
    
    \def\x{1.0 / 6}
    \def\y{1.0}
    \def\z{1.0}

    \coordinate (A) at (0,0);
    \coordinate (B) at (\x * 6, \x * 3);
    \coordinate (C) at (\x * 9, -\x);
    \coordinate (A1) at ($(A) + (0, 1.41 * \x * 3)$);
    \coordinate (B1) at ($(B) + (0, 1.41 * \x * 3)$);
    \coordinate (C1) at ($(C) + (0, 1.41 * \x * 3)$);


    \coordinate (K) at ($(B1) !2/3! (C1)$);
    \coordinate (N) at ($(A1) !2/3! (B1)$);
    \coordinate (M) at ($(A) !1/3! (B)$);
    \coordinate (E) at ($(C) !1/2! (C1)$);
    
    \coordinate (L) at ($(M) + (K) - (N)$);
    \coordinate (P) at ($(C) + 1/3*(C) - 1/3*(B)$);   
    
    

    

    \draw[dashed] (B) -- (C);
    \draw[dashed] (A) -- (B);
    \draw[dashed] (B) -- (B1);
    \draw[dashed] (P) -- (C);
    
    

    \draw[thick] (A) -- (C);
    \draw[thick] (C) -- (C1);
    \draw[thick] (A) -- (A1);
    \draw[thick] (B1) -- (A1);
    \draw[thick] (C1) -- (B1);
    \draw[thick] (E) -- (P);
    \draw[thick] (L) -- (P);


    \filldraw[fill=blue!15, draw=blue!0, thick, opacity=0.7] 
                (M) -- (N) -- (K) -- (E) -- (L) -- cycle;

    \draw[thick, blue] (K) -- (N);
    \draw[dashed, blue] (M) -- (N);
    \draw[dashed, blue] (M) -- (L);
    \draw[dashed, blue] (K) -- (E);
    \draw[thick, blue] (E) -- (L);

    
    
    

    \draw[thick] (C1) -- (A1);
    \draw[dashed] (K) -- (L);
    
   

    \node[below left, my label] at (A) {$A$};
    \node[below left, my label] at (C) {$C$};
    \node[right, my label] at (B) {$B$};
    \node[left, my label] at (A1) {$A_1$};
    \node[right, my label] at (C1) {$C_1$};
    \node[above, my label] at (B1) {$B_1$};

    \node[right, my label] at (K) {$K$};
    \node[above left, my label] at (N) {$N$};
    \node[above left, my label] at (M) {$M$};
    \node[below, my label] at (L) {$L$};
    \node[right, my label] at (P) {$P$}; 
    \node[right, my label] at (E) {$E$};
    
    
    \path (K) -- (C1) node[midway, above, sloped, my formula] {$1$};
    \path (C1) -- (E) node[midway, right, my formula] {$\frac{\sqrt{2}}{2}$};
    \path (E) -- (C) node[midway, left, my formula] {$\frac{\sqrt{2}}{2}$};    
    
    \foreach \point in {A,B,C,A1,B1,C1,N,K,M,L,P,E} {
      \node[vertex] at (\point) {};
    }
  \end{tikzpicture}
}
%--end figures imports--

% Форматирование заголовков глав
\titleformat{\chapter}[display]
     {\centering\bfseries} %выравнивание заголовка по центру
     {\MakeUppercase{\chaptertitlename} \thechapter}  % Заголовок "ГЛАВА 1"
     {0em}  % Отступ перед заголовком
     {\MakeUppercase} % Заголовок в верхнем регистре
     \titlespacing*{\chapter}{0pt}{3\baselineskip}{3\baselineskip}  % Отступы перед и после

% Форматирование секций
\titleformat{\section}
     {\normalsize\bfseries} %нормальный размер, полужирный шрифт
     {\thesection} % Нумерация секции
     {1em} % Отступ перед заголовком
     {} % Заголовок в верхнем регистре
\titlespacing*{\section}{1.25cm}{*1}{*1} % Отступы перед и после заголовка с абзацным отступом

% Форматирование подсекций
\titleformat{\subsection}
     {\normalsize\bfseries} 
     {\thesubsection} % Нумерация подсекций
     {1em} % перед заголовком
     {}
\titlespacing*{\subsection}{1.25cm}{*1}{*1} % Отступы перед и после с абзацным отступом
\usepackage{geometry}
\geometry{left=2cm}
\geometry{right=2cm}
\geometry{top=2cm}
\geometry{bottom=2cm}





\makeatletter
\renewenvironment{titlepage}{%
    \thispagestyle{empty} % убираем номер на титульной
    \setcounter{page}{1} % но считаем ее первой страницей
    \cleardoublepage

    
}{%
    
    \vfil\null
    \cleardoublepage
}
\makeatother
 
 \begin{document}





%ТИТУЛЬНАЯ СТРАНЦА
\begin{titlepage}

\begin{center}

\LARGE Неосократический диалог
\\Санкт-Петербург
\\2025
\end{center}

\end{titlepage}
\tableofcontents
%ВВЕДЕНИЕ
\pagestyle{plain}

\newpage

\chapter*{Введение в неосократический диалог}    



\addcontentsline{toc}{section}{Введение в неосократический диалог}
Важная задача школьного образования — развитие критического мышления учеников. Изучение геометрии несомненно способствует решению этой задачи, но у многих учащихся возникают трудности в понимании материала.  Рассмотрим, как можно достичь развития критического мышления и глубокого понимания материала на уроках стереометрии с помощью неосократического диалога.

Диалог Сократа — это форма разговора Сократа один на один с собеседником о неоднозначных вопросах с использованием опровержения утверждений для достижения истины. В диалоге «Менон», написанном Платоном, Сократ, не давая готовых знаний, подводит необразованного юношу к пониманию правил удвоения квадрата, предлагая построить то, что не существует. Этот пример показывает применение сократического подхода в познании конкретных предметов и развития пространственного воображения.
Есть несколько видов сократического диалога. Нельсон использовал данный метод в работе с младшими школьниками, Липман — в психологии. Суть сократического диалога заключается в правильно поставленных вопросах, направляющих собеседника к самостоятельному достижению истины.


Более эффективным способом развития критического мышления школьников является неосократический диалог. От сократического диалога неосократический отличается тем, что он рассчитан на тип взаимодействия учителя и класса, где учитель является модератором и воздерживается от любого выражения мнения. Используя лишь вопросы, педагог указывает на недостаточность аргументации ученика, провоцируя его найти противоречия в его ответе и посмотреть на задачу с другой стороны. Для того, чтобы подготовиться к неосократическому диалогу, учителю важно хорошо разбираться в теме, продумать основополагающие вопросы и разные направления диалога. Неосократический метод нацелен на определение сути изучаемого предмета. Он наиболее эффективен в ситуациях, когда ученик сталкивается с затруднениями, не знает, что ответить, приходит к противоречию. В этот момент благодаря принципу решения задачи через систему вопросов, предоставляя материал для самостоятельного рассуждения, учащийся приходит к решению проблемы.

Следует отметить, что мы говорим о неосократическом диалоге, поскольку в классическом сократическом методе отсутствует четкая конечная цель – достижение конкретного результата. В образовательном процессе же необходима направленность на достижение учебной цели.

\begin{center}
Принципы использования неосократического диалога:
\end{center}

\begin{itemize}

\item	Постановка неоднозначного вопроса
\item	Опора на жизненный опыт и знания учащихся
\item	Сопоставление мнений за счет поставленного в начале общего вопроса, не имеющего однозначного ответа
\item	Стимулирование появления гипотез у учеников за счет вопрошающей формы взаимодействия
\item	Ключевым принципом является прояснение проблемы, выявление ее сути перед переходом к частным случаям. Эффективным приемом является изменение материала задачи, демонстрация тех же закономерностей в иной плоскости, чтобы помочь ученику увидеть то, что ускользает в рамках конкретной темы. Для этого можно использовать простые темы, опирающиеся на жизненный опыт ученика, построив диалог таким образом, чтобы на их примере продемонстрировать общий метод рассуждения. Ученик должен не просто получить готовое решение, а научиться самостоятельно искать его.
\item	Сократический метод основан на принципе наведения. Важно, чтобы ученик сам пришел к пониманию, выдвигал гипотезы и, что особенно важно, научился применять этот метод к самообразованию. Учитель в данном подходе – не источник истины, а направляющая сила. Он не указывает на правильность или ошибочность решения, а лишь задает наводящие вопросы, выявляет несостыковки и противоречия в рассуждениях, чтобы ученик самостоятельно пришел к осознанию своего незнания. Только после осознания незнания учитель может предложить помощь в нахождении правильного пути.
\item	Важно подчеркнуть предметность, направленность на учебную цель и на абстрактный пример. Необходимо учитывать мнение собеседника, выстраивая диалог таким образом, чтобы каждый последующий вопрос логически вытекал из предыдущего ответа, постепенно подводя ученика к пониманию сути вопроса.
\item	Циклическая последовательность: общий вопрос — диалог — гипотеза — аргументация — вопрос — вызов противоречия — гипотеза
\item	Творчество мысли

\end{itemize}

\begin{center}
Этапы работы с помощью неосократического диалога:
\end{center}

\begin{enumerate}
    
\item	Постановка общего вопроса.
Вопрос должен не иметь однозначного ответа и быть достаточно обширным.
\item	Высказывания учеников по поводу темы.
Важно дать учащимся возможность высказать свою точку зрения в доброжелательной обстановке и без критики. При этом учитель может подчеркнуть положительные стороны мнений учеников, но без конкретного указания, что верно, а что нет.
\item	Указания на противоречия в ответе.
На данном этапе важно не указать учащимся на их ошибки, а позволить им самим додуматься, где у них противоречие в рассуждениях, и усомниться в собственных ответах.
\item	Достижение истины.

\end{enumerate}

Данный метод можно применять в разных случаях: в начале работы с задачей при анализе данных, условий, заключения; в момент выбора способов решения; на заключительном этапе в технологии варьирования. Так как применение неосократического диалога зависит от уровня знаний класса, темы материала, то время на использование данного метода может существенно отличаться. Поэтому при длительном исследовании одной задачи в заключение можно предложить учащимся вариации данной задачи с другими условиями. Благодаря глубокому пониманию задачи, ученики впоследствии смогут адаптироваться к изменениям данных. Решение простых задач, безусловно, не требует применения столь сложного метода, как сократический диалог. Однако, даже при изучении базовых понятий и отработке простых навыков, полезно внедрять элементы неосократического диалога, подготавливая учащихся к дальнейшей работе с этим методом в решении более сложных и открытых задач. Это позволит сформировать у учеников задатки критического мышления и научит их самостоятельно находить решения.

\chapter*{Отличия проблемного диалога от неосократического диалога}
\addcontentsline{toc}{section}{Отличия проблемного диалога от неосократического диалога}

Обратим внимание, что вопросы учителя при применении неосократического диалога могут быть проблемными, что может относить этот метод к проблемному диалогу. Основное отличие заключается в том, что в проблемном диалоге учитель может сообщать какую-то информацию, говорить в ту ли сторону движется ученик, возможно, оценивать его направление мысли. В сократическом диалоге это невозможно. В таблице приведены общие и различные черты двух методов.
\par 
\vspace{2\baselineskip}
\begin {small}
\begin{tabularx}{0.8\textwidth} { 
  | >{\centering\arraybackslash}X 
  | >{\centering\arraybackslash}X| }
 \hline
 \textbf{Проблемный диалог} & \textbf{Неосократический диалог}  \\
 \hline
 Конкретный вопрос  & Общий неоднозначный вопрос    \\
\hline
Чаще применяется на уроках при введении нового материала & В начале работы с задачей при анализе данных; в момент выбора способов решения; на заключительном этапе в технологии варьирования\\
\hline
Учитель может давать новую информацию, давать оценку ответам учеников & Учитель - модератор процесса, задающий вопросы, следующие друг за другом, и не оценивающий ответы учеников\\
\hline
Основной целью является получение новых знаний & Основной целью является поиск истины путем следования за вопросами учителя \\
 \hline
 Активное участие учеников & Активное участие учеников\\
\hline
\end{tabularx}
\end{small}
\par
\begin {small}
\begin{tabularx}{0.8\textwidth} { 
  | >{\centering\arraybackslash}X 
  | >{\centering\arraybackslash}X| }
 \hline
 \textbf{Проблемный диалог} & \textbf{Неосократический диалог}  \\

\hline
\begin{enumerate}
    \item Постановка проблемы
    \item Поиск решения
    \item Формулирование вывода
\end{enumerate} 
&
\begin{enumerate}
    \item Вопрос — ответ ученика
    \item Наведение на противоречие
    \item Осознание учеником ошибки
    \item Гипотеза
    \item Приход к истине
\end{enumerate} \\
\hline
Противоречие с собственными знаниями и не знанием чего-то, что не позволяет ему решить задачу прежними методами & Противоречие между выдвигаемым утверждением и здравому смыслу, которому оно противоречит\\

 \hline
\end{tabularx}
\end{small}
\par
\vspace{2\baselineskip}
Рассмотрим на примере решения задачи использование неосократического диалога и проблемного.

Постройте сечение правильной 4-угольной пирамиды ABDS плоскостью, проходящей через точки $M$ на $CS$ и $N$ на $DC$ и параллельно $AS$.

В задаче у учеников может получиться разный ответ, в зависимости от того, где именно были выбраны точки. Неосокротический диалог поможет разобрать сразу все варианты и понять их.
\begin{center}
\textbf{Сократический диалог}    
\end{center}



\textit{Учитель}: Ребята, Вам нужно построить такое сечение с такими условиями. Как следует выбрать точки $M$ и $N$?

\textit{Ученик}: По условию они лежат соответственно на $CS$ и $DS$,  поэтому можем взять как хотим, допустим так.

\textit{Учитель}: Так, принято.

% \includegraphics[]{p2.png}
\begin{center}
        \begin{minipage}{\textwidth}
            \centering
            \drawPyramidTask[]{5} 
            % \vspace{-0.3cm}
            % \hspace{-1cm}
            % \caption{}
        \end{minipage}
\end{center}

\textit{Учитель}: Так, идем дальше, какой будет следующий шаг?

\textit{Ученик}: Ну мы должны соединить $M$ и $N$, и через $M$ провести прямую в плоскости $ASC$ параллельно $AS$, и провести прямую через получившуюся точку $K$ и $N$ до пересечения с $AB$. Получили точку Т

\textit{Учитель}: Все сделали, можем теперь утверждать, что это сечение?

\textit{Ученик}: Да

\textit{Учитель}: А что же такое сечение многогранника?

\textit{Ученик}: Сечение многогранника - это многоугольник, вершины которого лежат на ребрах, а стороны на гранях.

\textit{Учитель}: Да, но получается, что сторона $ТМ$ лежит внутри пирамиды

\textit{Ученик}: Да, значит, это не сечение.

\textit{Ученик}: Нам надо, чтобы образовался еще один отрезок в этом сечении, содержащий точку $T$. Можем провести $ТF$ параллельно $AS$, и тогда все получится!

% \includegraphics[]{p3.png}

\begin{center}
    \begin{minipage}{\textwidth}
        \centering
        \drawPyramidTaskPTwo[]{5} 
        % \vspace{-0.3cm}
        % \hspace{-1cm}
        % \caption{}
    \end{minipage}
\end{center}

\textit{Учитель}: Так, интересно, действительно, теперь сечение полностью получилось. Что это за фигура?

\textit{Ученик}: Четырехугольник.

\textit{Учитель}: А всегда ли он будет получаться, если будем двигать точки $M$ и $N$?

\textit{Ученик}: Ну, наверное да, смысл останется тем же.

\textit{Учитель}: Так, давайте посмотрим такой вариант, где теперь точка $N$ будет лежать ближе к точке $D$.



% \includegraphics[]{p4.png}

\begin{center}
    \begin{minipage}{\textwidth}
        \centering
        \drawPyramidTaskPThree[]{5} 
        % \vspace{-0.3cm}
        % \hspace{-1cm}
        % \caption{}
    \end{minipage}
\end{center}

\textit{Учитель}: Как тут будем строить?

\textit{Ученик}: Аналогично предыдущему пункту мы должны соединить $M$ и $N$, и через $M$ провести прямую в плоскости $ASC$ параллельно $AS$, и провести прямую через получившуюся точку $K$ и $N$ до пересечения с $AB$. Получили точку $T$.

\textit{Учитель}: Так, но у меня почему-то прямая пересекает $BC$, а не $AB$.

\textit{Ученик}: Да, у меня тоже.

\textit{Учитель}: Допустим, что это так, как действуем дальше?

\textit{Ученик}: Ну как будто бы остается просто соединить точки $T$, $M$ и $N$ и наше сечение готово.


% \includegraphics[]{p5.png}
\begin{center}
    \begin{minipage}{\textwidth}
        \centering
        \drawPyramidTaskPFour[]{5} 
        % \vspace{-0.3cm}
        % \hspace{-1cm}
        % \caption{}
    \end{minipage}
\end{center}

\textit{Учитель}: Так, действительно, сечение построено. Оно удовлетворяет нашим условиям?

\textit{Ученик}: Да.

\textit{Учитель}: Но при этом это не четырехугольник, а треугольник. Как вы думаете, почему это возможно?

\textit{Ученик}: Наверное, в какой-то момент четырехугольник станет треугольником

\textit{Учитель}:Так, а в какой момент?

\textit{Ученик}: В тот момент, когда непонятно прямая будет пересекать либо $AB$, либо $BC$, т.е. можно рассмотреть пограничную точку $B$. Она принадлежит обеим сторонам.

\textit{Учитель}: Интересно, давайте попробуем.

\textit{Ученик}: Только теперь мы должны пойти с конца. Провести отрезок $ВN$, от точки $K$ отложить $KM$ параллельно $АS$, и получится точка $M$. И сечение $BMN$

% \includegraphics[]{p6.png}
\begin{center}
    \begin{minipage}{\textwidth}
        \centering
        \drawPyramidTaskPFive[]{5} 
        % \vspace{-0.3cm}
        % \hspace{-1cm}
        % \caption{}
    \end{minipage}
\end{center}

\textit{Учитель}: Как интересно, а ведь правда так и получается, т.е. какой мы можем сделать вывод насчет формы сечения в данной задаче?

\textit{Ученик}: Что оно может меняться в зависимости от того, как мы возьмем точки на ребрах

\textit{Учитель}: Да, верно, именно так! Мы с вами двигали точку $N$, а что будет, если двигать точку $M$?

\textit{Ученик}: Будут аналогичные рассуждения, т.к. отрезок $NT$ все равно упадет либо на $AB$ либо на $BC$, и получатся так же либо четырехугольник в сечении, либо треугольник. А из движения точки $M$ фигура будет становиться выше или ниже.

\textit{Учитель}: Да, все так!

\begin{center}
    \textbf{Проблемный диалог}
\end{center}



\textit{Учитель}: Ребята, нам нужно построить сечение пирамиды, как нам это сделать?

\textit{Ученик}: Нужно соединить $M$ и $N$, провести через $M$ прямую в плоскости $ASC$ параллельно $AS$, найти точку $K$, затем провести $KN$ до пересечения с $AB$ — получим точку $T$. Сечение — это $TMN$.

\textit{Учитель}: Хорошо, а если я скажу, что $MN$ лежит внутри пирамиды, а не на её грани? Будет ли TMN сечением?

\textit{Ученик}: Ну... да? Раз мы соединили точки на рёбрах...

\textit{Учитель}: Но сечение — это многоугольник, все стороны которого лежат на гранях, а вершины — на рёбрах. $MN$ лежит внутри, значит, это не сторона сечения. Как тогда достроить фигуру, чтобы получилось правильное сечение?

\textit{Ученик}: Может, нужно провести ещё одну прямую? Например, через $T$ параллельно $AS$?

\textit{Учитель}: Да, верно. Почему именно параллельно $AS$? Как это поможет? И точно ли получится четырёхугольник? А если точки $M$ и $N$ будут в других местах — что тогда?

\textit{Ученик}: (задумывается, начинает анализировать возможные варианты) 

\textit{Учитель}:Попробуйте дома рассмотреть разные случаи расположения $M$ и $N$ и выяснить, когда сечение будет треугольником, а когда — четырёхугольником.

\chapter*{Особенности составления неосократичского диалога}
\addcontentsline{toc}{section}{Особенности составления неосократичского диалога}

Как было отмечено ранее, неосократический диалог как методический инструмент отличается объективностью, безоценочностью вопросов модератора (в нашем случае, учителя), неоднозначностью первого вопроса, задающего ход всей беседы. Выделим основные правила составления вопросов, подходящих для использования неосократический диалог на уроке математики:



\begin{enumerate}
    
\item	Принципиальное отсутствие однозначного и единственного ответа на вопрос,  который задает направление беседы
\item	Лаконичность и четкость формулировки вопроса.
\item	Последовательность и систематичность вопросов; каждый новый вопрос следует из ответа на предыдущий вопрос
\item Содержание в вопросе альтернативной точки зрения, помогающий найти противоречие или новую идею.
\item  Содержание в вопросах основных посылок учителя, необходимых для направления беседы.

\end{enumerate}

На первом этапе постановки общего вопроса можно использовать универсальную формулировку: «Как решить поставленную задачу?» Но есть вероятность столкнуться с очевидным ответом: «Мы не знаем». Поэтому следует выбирать более конкретные формулировки, например: «Какие методы можно использовать при решении этой задачи?», «В чём сложность задачи?» или «Какой результат может получиться?» (в частности, при работе с сечениями).

Далее учителю необходимо выслушать предположения учащихся, важно и полезно поддерживать их диалог между друг другом, вмешиваясь для поддержания направления исследования. При остановке их рассуждений можно помочь учащимся, намекнув вопросом на противоречие или подсказав перспективную траекторию мысли: «Все ли данные условия проанализированы?», «Каких сведений не достаточно?». 

Если найдено конкретное противоречие с теорией, необходимо обратить на это внимание, но не указывать путь его разрешения, к примеру: «Может ли эта плоскость проходить через эту точку?», «Может ли эта прямая быть перпендикулярной к данной плоскости?».

При необходимости всё таки привести учащихся к конкретным мыслям неосократический метод предполагает использование закрытых вопросов, содержащих посылки и требующих согласия или несогласия собеседника. Например: «Правда ли, что квадрат гипотенузы равен сумме квадратов катетов?», «Верно ли, что плоскость однозначно задаётся тремя точками?». Такими вопросами учитель сможет привести учащихся к нужным ему выводам, но при этом ограничит их свободные рассуждения. Это необходимо в рамках урока из-за ограниченности времени, зато даёт учащимся примеры правильных вопросов, которые они должны научиться задавать самим себе для того, чтобы потом самостоятельно решать задачи.

На этапе «достижения истины» важно зафиксировать успех и обратить на него внимание учащихся. Успех может быть, когда мы решили одну подзадачу, так и тогда, когда достигнуто полное решение и проведён анализ. Анализ является важной частью решения задачи, необходимо рассмотреть всё решение, применяемые методы и  варианты развития задачи. Могут использоваться следующие вопросы:

\begin{enumerate}
\item Какой путь решения оказался самым эффективным? Почему другие пути не помогли прийти к результату? 
\item Как повлияет изменение конкретных условий на решение задачи?
\item Как можно развить задачу? Что ещё можно найти и доказать?
\end{enumerate}

Метод неосократического диалога можно применять по-разному в зависимости от уровня и мотивированности класса. Если работа ведется в «сильном» классе, где учащиеся легко ведут диалог, выдвигают гипотезу и хорошо справляются с разнообразными задачами, то некоторые шаги решения могут сократиться, потому что дети быстрее придут к правильным утверждениям. Зато вопросы с большим исследовательским потенциалом в таком классе можно рассматривать более подробно. Тогда и задачи для ведения беседы следует выбирать более сложные, подойдут также частично-открытые задачи, которые предусматривают несколько ответов. В таком случае учащиеся смогут обсудить разные пути решения, то, от чего зависит результат, выделить закономерности и обобщить полученные знания.

В классах, где наблюдаются трудности с усвоением материала и построением беседы, рассматриваемая методика будет полезна в качестве средства развития математической грамотности, умения рассуждать и критически мыслить. Учитель следует выбирать не сложные задачи и вместе с классом последовательно в ходе беседы идти к её решению. Необходимо использовать большое число закрытых вопросов, но по возможности следует всегда давать учащимся возможность порассуждать и высказать свои идеи, даже если они все не приведут в конечном итоге к решению задачи. Учащиеся должны получить опыт неудачи и научиться справляться с ней, опровергать свои гипотезы, возвращаться к предыдущим утверждениям и выдвигать новые. 

Таким образом, метод неосократического диалога может быть каждым учителем адаптирован к ситуации в конкретном классе для достижения разнообразных целей, таких как научить учащихся рассуждать, логически мыслить, высказывать свои гипотезы и не бояться совершить ошибку, научить исследовать геометрические конструкции и видеть разные вариации одной и той же конструкции.

Приведем некоторые шаблоны вопросов для неосократического диалога, которые помогут соблюсти стилистику и регулировать течение диалога
\begin{enumerate}
\item Какие методы можно использовать при решении этой задачи?? 
\item В чём сложность задачи?
\item Какой результат может получиться?
\item Не правда ли / Скажите мне / Согласны ли вы, что… [посылка]?
\end{enumerate}
\chapter*{Решение задач методом неосократичского диалога}
\addcontentsline{toc}{section}{Решение задач методом неосократичского диалога}

Рассмотрим некоторые примеры неосократического диалога при решении стереометических задач. Первая задача состоит в исследовании возможных сечений куба плоскостью. Решение этой задачи способствует развитию исследовательских навыков школьников, позволяет получить опыт самостоятельного получения теоретических сведений. 


\textbf{Пример 1.}

 \textit{Учитель}: Добрый день, класс. Сегодня мы будем с вами исследовать куб. Каким многоугольником может быть сечение куба плоскостью? 
 
 \textit{Ставим общий неоднозначный вопрос.}
 
 \textit{Ученики}: Треугольник, четырехугольник, пятиугольник и т.д.

 \textit{Учитель}: Отлично! Изобразите в тетрадях сечение куба, в котором получается треугольник. (Рис. 1)

 \begin{center}
    \begin{minipage}{0.3\textwidth}
        \centering
        \drawCubeTaskTwo[]{3} 
        % \vspace{-0.3cm}
        % \hspace{-1cm}
        % \caption{}
    \end{minipage}
  \hspace{-0.2cm}
    \begin{minipage}{0.3\textwidth}
        \centering
        \drawCubeTaskTwoPTwo[]{3} 
        % \vspace{-0.3cm}
        % \hspace{-1cm}
        % \caption{}
    \end{minipage}
      \hspace{-0.2cm}
    \begin{minipage}{0.3\textwidth}
        \centering
        \drawCubeTaskTwoPThree[]{3} 
        % \vspace{-0.3cm}
        % \hspace{-1cm}
    \end{minipage}
    \capt{1}
\end{center}

\textit{Учитель}: Какие треугольники у вас получились? А у кого-нибудь получился прямоугольный треугольник? А тупоугольный? 

\textit{Ученик 1}: У меня получился тупоугольный. 

\textit{Ученик 2}: У меня прямоугольный, кажется. 

\textit{Ученик 3}: Я не могу понять, какой у меня треугольник.

\textit{Учитель}: Не правда ли, чтобы говорить о типе треугольника, необходимо доказать, что построенное сечение является именно остроугольным, прямоугольным или тупоугольным треугольником?

\textit{Ученики}: Правда.

\textit{Учитель}: А правда ли то, что верно для общего случая, верно и для любого частного?

\textit{Ученики}: Конечно.

\textit{Учитель}: А поможет нам выделение общих свойств треугольных сечений куба?

\textit{Ученики}: Да, поможет.

\textit{Учитель}: От чего зависит тип треугольника?

\textit{Ученики}: От углов треугольника. Но про углы в условии задачи ничего не сказано.

\textit{Учитель}:А если не получается сразу найти углы, что вспомогательное можно ещё найти?

\textit{Ученики}: Например, стороны треугольника.

\textit{Учитель}: Хорошо. Не правда ли все грани куба квадраты?

\textit{Ученики}: Верно.

\textit{Учитель}: Является ли треугольник $A'B'C'$ сечением куба (Рис. 2)? 

\textit{В этот момент можно вызвать одного из учеников к доске и попросить повторить его рисунок.}

 \begin{center}
    \begin{minipage}{0.3\textwidth}
        \centering
        \drawCubeTaskTwoPFour[]{3} 
        % \vspace{-0.3cm}
        % \hspace{-1cm}
    \end{minipage}
    \capt{2}
\end{center}

\textit{Ученики}: Является.

\textit{Учитель}: Тогда, если стороны сечения принадлежат граням куба, то есть квадратам, они являются сторонами прямоугольных треугольников?

\textit{Ученики}: Конечно!

\textit{Учитель}: А какую основную теорему о прямоугольных треугольниках вы знаете?

\textit{Ученики}: Теорему Пифагора.

\textit{Учитель}: Как звучит эта теорема?

\textit{Ученики}: Квадрат гипотенузы будет равен сумме квадратов катетов.

\textit{Учитель}: Как можно применить тогда теорему Пифагора для решения нашей задачи?

\textit{Ученик 1}: Стороны треугольника также являются сторонами прямоугольных треугольников.

\textit{Ученик 2}: Тогда можно обозначить общие для новых треугольников отрезки буквами и выразить стороны исходного треугольника через отрезки ребер квадрата! Обозначим  $A_1D'$ за $a$, $A_1A'$  за $b$, $A_1B'$  за $c$.

\textit{Ученик 3}: Тогда $A'B'^{\ 2}=b^2+c^2, A'D'^{\ 2}=a^2+b^2, D'B'^{\ 2}=a^2+c^2$. 

\textit{Эти записи следует продублировать на доске.}

\textit{Учитель}: Теперь можем подойти к исследованию углов нашего треугольника. Какая теорема поможет нам при помощи сторон треугольника определить тип его углов?

\textit{Ученики}: Теорема синусов.

\textit{Идут по ложному пути, учитель не отвергает идею, но обращает внимание на противоречие.}

\textit{Учитель}: Не правда ли, что мы имеем квадраты сторон, но не сами стороны?

\textit{Ученики}: Правда.

\textit{Учитель}: Будет ли тогда удобно работать с теоремой синусов?

\textit{Ученики}: Действительно, придётся применять радикалы, возможно,  мы не сможем получить результат.

\textit{Учитель}: А вспомните, какой знак имеют синус острого и синус тупого угла?

\textit{Ученики}: Они оба положительные.

\textit{Учитель}: Не правда ли, тогда синус не сможет дать нам ответ на вопрос о типе угла?

\textit{Ученики}: Да, это так. Тогда можно применить теорему косинусов!

\textit{Ученики}: Да. Тогда выразим квадрат одной из сторон нашего треугольника, например $A'D'$,  по теореме косинусов и определим знак косинуса противоположного ей угла.

\textit{Кто-то из учеников выходит выписывать выражения на доске:}

\begin{enumerate}
    \item $A'D'^{\ 2}= A'B'^{\ 2} + B'D'^{\ 2} - 2\cdot A'B'\cdot B'D'\cdot cos\angle A'B'D'$
    \item $a^2 + b^2 = b^2 + c^2 + a^2 + c^2 - 2 \cdot(b^2 + c^2)\cdot(a^2 + c^2)\cdot cos\angle A'B'D'$
    \item $2 \cdot(b^2 + c^2)\cdot(a^2 + c^2)\cdot cos\angle A'B'D' = 2c^2\Rightarrow cos\angle A'B'D' > 0$
\end{enumerate}

\textit{Учитель}: Что значит, если косинус угла $\angle A'B'D'$ положителен?

\textit{Ученики}: Тогда этот угол острый.

\textit{Учитель}: Отлично! Будет ли разница, для какого угла применять теорему косинусов?

\textit{Ученики}: Нет, результат будет аналогичный.

\textit{Учитель}: Замечательно. Значит, мы можем рассмотреть только один угол и на его основе сделать необходимые выводы?

\textit{Ученики}: Да, можем. Тогда все углы острые, а значит треугольник остроугольный!

\textit{Учитель}: Мы с вами рассматривали общий случай? Верен ли наш вывод для всех треугольных сечений куба?

\textit{Ученики} Да, верен. Тогда любое треугольное сечение куба есть остроугольный треугольник.

\textit{Учитель}: Молодцы! Запишем это как вывод.

\textit{Предтавленный выше отрывок иллюстрирует ситуацию в классе, при которой учащиеся не готовы самостоятельно вести беседу на уроке, поэтому учителю приходится задавать много вопросов, содержащих основные посылки, для получения решения задачи. Но так ученики получают опыт и пример рассуждений, видят причинно-следственные связи. Со временем, когда учащиеся привыкнут к такому формату работы и обретут необходимые навыки, они смогут самостоятельно вести  разговор.}

\textit{Учитель}: Не правда ли, мы узнали, каким может быть треугольное сечение куба? Какие сечения следует рассмотреть далее?

\textit{Ученики}: Это так. Теперь можно рассмотреть четырехугольные сечения.

\textit{Учитель}: Хорошо. Тогда как вы думаете, какие четырехугольники могут получиться в сечении многогранника?

\textit{Ученики}: Квадрат, прямоугольник, ромб, параллелограмм, трапеция…

\textit{Учитель}: Можете ли вы построить сечения, которое является параллелограммом, прямоугольником, квадратом, ромбом, трапецией?

\textit{Ученики}: Можем (Рис. 3).

 \begin{center}
    \begin{minipage}{0.3\textwidth}
        \centering
        \drawCubeTaskTwoPFive[]{3} 
        % \vspace{-0.3cm}
        % \hspace{-1cm}
        % \caption{}
    \end{minipage}
  \hspace{-0.2cm}
    \begin{minipage}{0.3\textwidth}
        \centering
        \drawCubeTaskTwoPSix[]{3} 
        % \vspace{-0.3cm}
        % \hspace{-1cm}
        % \caption{}
    \end{minipage}
      \hspace{-0.2cm}
    \begin{minipage}{0.3\textwidth}
        \centering
        \drawCubeTaskTwoPSeven[]{3} 
        % \vspace{-0.3cm}
        % \hspace{-1cm}
    \end{minipage}
       \hspace{-0.2cm}
    \begin{minipage}{0.3\textwidth}
        \centering
        \drawCubeTaskTwoPEight[]{3} 
        % \vspace{-0.3cm}
        % \hspace{-1cm}
    \end{minipage}
       \hspace{-0.2cm}
    \begin{minipage}{0.3\textwidth}
        \centering
        \drawCubeTaskTwoPNine[]{3} 
        % \vspace{-0.3cm}
        % \hspace{-1cm}
    \end{minipage}
    \capt{3}
\end{center}

\textit{Возможно подробно обсудить построения параллелограмма, прямоугольника, квадрата и ромба в сечении куба. Это следует сделать по образцу работы с треугольными сечениями. Мы это обсуждение опускаем, останавливаясь на типах трапеций.}

\textit{Учитель}: Хорошо. Но ведь существуют разные типы трапеций, не так ли?

\textit{Ученики}: Совершенно верно. Трапеции делятся на произвольные, прямоугольные и равнобедренные.

\textit{Учитель}: А как вы думаете, как построить сечение так, чтобы оно являлось равнобедренной трапецией?

\textit{Ученик 1}: Во-первых, основания трапеции должны располагаться в параллельных гранях куба.

\textit{Ученик 2}: Во-вторых, боковые стороны должны быть равны.

\textit{По сравнению с предыдущим отрывком про треугольные сечения, в этом отрывке диалога учащиеся проявляют большую инициативу. Учителю уже не приходится вставлять в свои вопросы все посылки, он опирается на гипотезы учеников и помогает им прийти своим путем к решению задачи.}

\textit{Учитель}: А как сделать боковые стороны равными?

\textit{Ученик 1}: Боковые стороны располагаются в смежных гранях.

\textit{Ученики 2}: А ещё они вместе с боковыми ребрами куба образуют в боковых гранях новые прямоугольные трапеции!

\textit{Ученик 3}: Тогда эти прямоугольные трапеции должны быть равны.

\textit{Ученик 1}: А значит основания трапеции в своих гранях должны образовывать прямоугольные равнобедренные треугольники.

\textit{Учитель}: То есть, стоит правильно выбрать именно основания трапеции?

\textit{Ученик 2}: Да, основания трапеции должны быть параллельны диагонали квадрата, являющегося гранью куба, чтобы треугольники были равнобедренными.

\textit{Учитель}: Правильно ли, что можно рассмотреть четырехугольник образованный двумя параллельными  диагоналями, например, верхней и нижней граней?

\textit{Учителю следует использовать в том числе вопросы, на которые подразумевается отрицательный ответ, чтобы тренировать внимание и самостоятельность учащихся. Это позволит сфокусировать их на диалоге, не даст им все время полагаться на подсказки учителя.}

\textit{Ученик 3}: Нет, нельзя. Эти диагонали будут равны, тогда по признаку параллелограмма мы получим не трапецию, а параллелограмм.

\textit{Ученик 2}: Так ведь это куб, значит мы получим вообще прямоугольник.

\textit{Учитель}: Тогда какие отрезки следует выбрать для оснований трапеции?

\textit{Ученики 2}: Основаниями трапеции должны служить отрезки в параллельных гранях куба, параллельные диагоналям соответствующих квадратов, но не равные между собой.

\textit{Учитель}: Кто попробует такое сечение построить (Рис. 4)? 

\textit{Кто-то выходит к доске.}

 \begin{center}
    \begin{minipage}{0.3\textwidth}
        \centering
        \drawCubeTaskTwoPTen[]{3} 
        % \vspace{-0.3cm}
        % \hspace{-1cm}
        % \caption{}
    \end{minipage}
    \capt{4}
\end{center}

\textit{Учитель}: С равнобедренной трапецией разобрались. Как построить прямоугольную трапецию в сечении? 

\textit{Ведём учеников к совершению ошибки.}

\textit{Ученик 1}: Снова основания трапеции должны быть в параллельных плоскостях.

\textit{Ученик 2}: Но теперь одна боковая сторона должна быть перпендикулярна основаниям. Значит, эта боковая сторона будет параллельная ребру куба, которое соединяет грани, в которых лежат основания трапеции.

\textit{Учитель}: Принимаем. Давайте изобразим это на рисунке (Рис. 5). Например, пусть $B_2C_2$ - эта боковая сторона трапеции $A_2B_2C_2D_2$. А какую фигуру в грани куба образует $B_2C_2$, если она параллельна боковому ребру?


 \begin{center}
    \begin{minipage}{0.3\textwidth}
        \centering
        \drawCubeTaskTwoPEleven[]{3} 
        % \vspace{-0.3cm}
        % \hspace{-1cm}
        % \caption{}
    \end{minipage}
    \capt{5}
\end{center}

\textit{Ученик 2}: Так как сторона трапеции параллельна ребру, то она делит грань на два прямоугольника.

\textit{Учитель}: Что из этого следует?

\textit{Ученик 1}: Что сторона трапеции отсекает равные отрезки от нижнего и верхнего ребер куба.

\textit{Учитель}: Правда ли, что плоскость пересекает параллельные плоскости по параллельным прямым?

\textit{Ученик 3}: Действительно. Тогда $D_2C_2$ параллельна $A_2B_2$.

\textit{Учитель}: А каким ребрам принадлежат точки $A_2$ и $D_2$?

\textit{Ученик 1}:  Ребрам $AB$ и $A_1B_1$ (Рис. 6).

\textit{Рассматривают только один случай. Он верный, но не единственный. Необходимо вернуться к упущенной из виду информации}

 \begin{center}
    \begin{minipage}{0.3\textwidth}
        \centering
        \drawCubeTaskTwoPTwelve[]{3} 
        % \vspace{-0.3cm}
        % \hspace{-1cm}
        % \caption{}
    \end{minipage}
    \capt{6}
\end{center}

\textit{Ученик 2}: Тогда треугольники в верхней и нижней гранях подобны?

\textit{Ученик 1}: Не просто подобны, они равны. Ведь отрезки $B_1C_2$ и $BB_2$ равны.

\textit{Ученик 3}: Тогда по признаку параллелограмма $A_2D_2B_1B$ - параллелограмм, но в нём два прямых угла уже есть, значит это прямоугольник.

\textit{Ученик 1}: Тогда и построенный четырехугольник не трапеция, а тоже прямоугольник. Что же, тогда нельзя построить сечение так, чтобы получилась прямоугольная трапеция?

\textit{Учитель}: А только ли таким образом может в кубе располагаться четырехугольное сечение?

\textit{Ученик 3}: Действительно, мы забыли, что если $ВС$ параллельно боковому ребру, то точка $A_2$ может лежать не только на ребре, перпендикулярном BC, но и на параллельной стороне. То есть нужно рассмотреть случай, когда точка $A_2$ принадлежит стороне $AD$ (Рис. 7).



 \begin{center}
    \begin{minipage}{0.3\textwidth}
        \centering
        \drawCubeTaskTwoPThirteen[]{3} 
        % \vspace{-0.3cm}
        % \hspace{-1cm}
        % \caption{}
    \end{minipage}
    \capt{7}
\end{center}

\textit{Учитель}: Тогда в каких гранях будут лежать стороны трапеции?

\textit{Ученик 2}: В четырех попарно параллельных гранях. Значит снова вторая боковая сторона сечения окажется параллельна боковому ребру куба, а значит наше сечение прямоугольник.

\textit{Учитель}: Возможны ли ещё какие-то варианты расположения сечения?

\textit{Ученики}: Нет, других вариантов нет.

\textit{Учитель}: Какой тогда можно сделать вывод?

\textit{Ученики}: Прямоугольная трапеция не может быть сечением куба.

\textit{Работа с четырехугольными сечениями в примере уже велась с большей включенностью и самостоятельностью учащихся, при этом учитель сохранял и указывал правильное направление, не давал сойти с курса.}

\textit{Учитель}: Мы исследовали четырехугольные сечения. Могут ли они быть ещё какими-то фигурами?

\textit{Ученики}:Сечение может быть пятиугольником.

\textit{Учитель}: Как должна располагаться плоскость в пространстве относительно куба, чтобы в сечении получился пятиугольник?

\textit{Ученик 2}: Она точно не должна быть параллельна ни одной грани, тогда выйдет в сечении квадрат.

\textit{Ученик 1}: И ни одной грани не должна быть перпендикулярна, тогда выйдет прямоугольник.

\textit{В этот момент учащиеся анализируют свой полученный ранее опыт и на его основе делают выводы и выдвигают новые гипотезы. Идет развитие способности рассуждать и строить логические связи.}

\textit{Ученик 3}: Тогда плоскость надо, как бы, наклонить, чтобы увеличить число углов в сечении. Но плоскость должна будет при этом проходить только через одно из двух оснований.

\textit{Ученик 2}: Да, иначе будет даже шестиугольник.

\textit{Учитель}: Не правда ли, вы уже нашли способ построения шестиугольного сечения куба?

\textit{Ученик 1}: Да, следует взять плоскость не параллельную и не перпендикулярную ни одной грани, но проходящую через все грани куба.

\textit{Учитель}: Хорошо, вернемся к этому позднее. 

\textit{Направляем диалог в нужное нам русло.}

\textit{Учитель}:  Как задать точками плоскость, необходимую для построения пятиугольного сечения?

\textit{Ученик 1}: Можно взять точку $K$ на боковом ребре куба, и ещё две $M$ и $N$ на соседних боковых ребрах так, чтобы два отрезка, образованные этими точками, не были параллельны ребрам оснований. 

\textit{Можно попросить ученика выйти и изобразить эти точки на доске}

\textit{Ученик 2}: Этого не достаточно, плоскость нужно правильно «направить». Точки $M$ и $N$ должна быть вместе ниже или выше первой точки $K$ (Рис. 8).

 \begin{center}
    \begin{minipage}{0.3\textwidth}
        \centering
        \drawCubeTaskTwoPFourteen[]{3} 
        % \vspace{-0.3cm}
        % \hspace{-1cm}
        % \caption{}
    \end{minipage}
    \capt{8}
\end{center}

\textit{Учитель}: То есть, я могу взять для сечения и точки основания?

\textit{Ученик 3}: Нет, их брать нельзя, в этом случае в сечении получится треугольник.

\textit{Учитель}: Как же точно определить наши точки?

\textit{Ученик 3}: Нужно взять три точки на соседних боковых ребрах куба, не совпадающие с его вершинами, так, чтобы обе крайние точки находились на меньшем расстоянии от одного основания, чем серединная точка.

\textit{Учитель}: Хорошо. Изобразим на доске куб. Отмечаем три точки. Соответствуют ли мои три точки вашим требованиям (Рис. 9)? 

    \textit{Строим две крайние точки немногим ниже точки}

\begin{center}
    \begin{minipage}{0.3\textwidth}
        \centering
        \drawCubeTaskTwoPFifteen[]{3} 
        % \vspace{-0.3cm}
        % \hspace{-1cm}
        % \caption{}
    \end{minipage}
    \capt{9}
\end{center}



\textit{Ученики}: Да, соответствуют.

\textit{Учитель}: И в сечении у нас получится пятиугольник.

\textit{Ученики}: Да, получится.

\textit{Строим, получается четырехугольник.}

\textit{Учитель} : Не правда ли, вышел четырехугольник, а не пятиугольник?

\textit{Ученики}: Правда, значит мы что-то не учли. Нужны ещё требования к точкам.

\textit{Учитель}: Удобный ли способ построения мы выбрали, если столько требований необходимо предъявлять к трем первым точкам?

\textit{Ученики}: Нет, неудобный. Надо выбрать другой путь.

\textit{Учитель}: Если не получается задать точки, с чего можно начать?

\textit{Ученик 1}: Можно работать с прямыми! Параллельные грани плоскость пересекает по параллельным прямым. Тогда можно задать прямую в одной грани и получить точки в параллельной грани.

\textit{Учитель}: Каким тогда образом должны располагаться прямые в параллельных гранях?

\textit{Ученик 3}: Нельзя, чтобы в сечении снова получился четырехугольник.

\textit{Ученик 2}: Отрезок в грани может делить её на треугольник и четырехугольник или на два четырехугольника. Нам нужно, чтобы в двух гранях получались разные случаи.

\textit{Учитель}: Правильно ли, тогда следует выбрать в параллельных гранях два параллельных отрезка, таких что один отсекает от грани треугольник, а другой делит грань на два четырехугольника?

\textit{Здесь учиель помогает сформулировать выввод из утверждений учащихся, чтобы они н запутались и были услышаны все высказывавшиеся.}

\textit{Ученик 3}: Да, только снова точки отрезков не должны совпадать с вершинами куба.

\textit{Учитель}: Тогда попробуйте построить два таких отрезка и сечение по ним. Получилось?

\textit{Ученики}: Да (Рис. 10).

\begin{center}
    \begin{minipage}{0.3\textwidth}
        \centering
        \drawCubeTaskTwoPSixteen[]{3} 
        % \vspace{-0.3cm}
        % \hspace{-1cm}
        % \caption{}
    \end{minipage}
    \capt{10}
\end{center}

\textit{Учитель}: Вы смогли построить пятиугольник. Единственный ли это способ?

\textit{Ученики}: Нет, можно продолжить рассуждения про точки, но такое построение будет более громоздким.

\textit{Учитель}: Но один из способов мы с вами рассмотрели. Попробуйте дома найти другие способы, и если ваш окажется красивее и лаконичнее этого, рассмотрим его на следующем занятии. А какой особый вид пятиугольника вы знаете?

\textit{Ученики}: Правильный пятиугольник. 

\textit{Учитель}: А может ли правильный пятиугольник быть сечением куба.

\textit{Ученик 2}: У правильного пятиугольника стороны равны, и у куба ребра равные, почему тогда нет.

\textit{Учитель}: А что ещё известно про правильный пятиугольник?

\textit{Ученик 1}: Все его углы равны 108 градусам.

\textit{Учитель}: Могут ли его стороны быть параллельными?

\textit{Ученик 1}: Нет, никак не могут.

\textit{Учитель}: А как плоскость пересекает параллельные грани куба?

\textit{Ученик 2}: По параллельным прямым. У пятиугольника пять сторон, а у куба шесть граней, значит две пары сторон точно будут в параллельных гранях, а значит должны быть параллельны.

\textit{Ученик 3}: Тогда правильный пятиугольник не может быть сечением куба, так как его стороны не могут быть параллельными.

\textit{Учитель}: С пятиугольником разобрались. Какое сечение мы отложили на потом?

\textit{Ученики}: Шестиугольное.

\textit{Учитель}: Вы уже говорили, как построить шестиугольник в сечении куба. Как же это сделать?

\textit{Ученик 3}: Следует взять плоскость не параллельную и не перпендикулярную ни одной грани, но проходящую через все грани куба.

\textit{Учитель}: Правда ли, что на моем рисунке секущая плоскость не параллельна и не перпендикулярна ни одной грани куба (Рис. 11)?

\begin{center}
    \begin{minipage}{0.3\textwidth}
        \centering
        \drawCubeTaskTwoPSeventeen[]{3} 
        % \vspace{-0.3cm}
        % \hspace{-1cm}
        % \caption{}
    \end{minipage}
    \capt{11}
\end{center}

\textit{Ученики}: Правда.

\textit{Учитель}: А какое требование не выполнено?

\textit{Ученики}: Плоскость не проходит через все грани куба.

\textit{Учитель}: Что нужно поменять, чтобы получить шестиугольник в сечении?

\textit{Ученик 1}: Можно «сдвинуть» верхний отрезок к противоположному углу, тогда плоскость больше наклонится, и пересечет другие грани.

\textit{Учитель}: А до какого момента нужно «двигать» верхний отрезок?

\textit{Ученик 3}: Границей будет диагональ верхнего квадрата. Если она принадлежит сечению, то оно является всё ещё четырехугольником. Если отрезок расположен за диагональю, то сечение будет уже шестиугольником.

\textit{Учитель}: Попробуйте построить это. Шестиугольник получился (Рис. 12). Может ли сечение быть правильным шестиугольником?

\begin{center}
    \begin{minipage}{0.3\textwidth}
        \centering
        \drawCubeTaskTwoPEighteen[]{3} 
        % \vspace{-0.3cm}
        % \hspace{-1cm}
        % \caption{}
    \end{minipage}
    \capt{12}
\end{center}

\textit{Ученик 2}: Для этого все стороны шестиугольника должны быть равны.

\textit{Учитель}: А какие фигуры образуют стороны шестиугольника в своих гранях?

\textit{Ученики}: Прямоугольные треугольники.

\textit{Учитель}: Какими сторонами этих треугольников будут стороны шестиугольника?

\textit{Ученики}: Они все будут гипотенузами.

\textit{Учитель}: Как же добиться того, чтобы все стороны шестиугольника были равными? 

\textit{Ученик 2}: Эти прямоугольные треугольники должна быть равными.

\textit{Ученик 3}: Тогда и их катеты должны быть соответственно равны, а значит вершины шестиугольника должны делить свои ребра одинаковым образом.

\textit{Ученик 1}: Тогда можно взять произвольное отношение отрезков и последовательно разделить ребра в этом отношении. Делящие точки и будут вершинами квадрата. 

\textit{Учитель}: Будет ли сохраняться требование параллельности противоположных сторон в таком случае?

\textit{Ученик 2}: Да, можно рассмотреть выносной рисунок и перенести на одну грань треугольник с противоположной грани. Тогда можно доказать параллельность гипотенуз с помощью равенства углов при пересечении двух прямых третьей прямой.

\textit{Учитель}: А какой особый случай такого сечения можно рассмотреть?

\textit{Ученики}: Когда делящие точки будут серединами ребер куба. Тогда все указанные прямоугольные треугольники будут равнобедренными и равными.

\textit{Учитель}: Отлично, нашли частный красивый случай. Изобразите это сечение.

\textit{Учитель}: С шестиугольником закончили, сможете построить семиугольное сечение?

\textit{Так как учитель не спрашивает, можно ли в принципе построить семиугольное сечение, то он ведет их к ошибочным рассуждениям. Но при подготовленности учеников, они сразу не согласятся с траекторией учителя, а будут размышлять и найдут несостыковки.}

\textit{Ученик 2}: Тогда семь сторон должно лежать в гранях куба.

\textit{Ученик 3}: Но в одной грани может лежать только одна сторона сечения.

\textit{Ученик 1}: Тогда нельзя построить больше шести сторон сечения, так как в кубе всего шесть граней.

\textit{Ученик 3}: И значит, семиугольник не может уже быть сечением куба.

\textit{Учитель}: А каким может быть более общий вывод?

\textit{В ходе диалога учащиеся рассмотрели все возможные типы сечений куба и на основе полученных знаний могут сделать общий вывод. Это иллюстрация того, как метод неосократического диалога позволяет организовывать исследовательскую работы в классе и учить детей размышлять и анализировать геометрические конструкции.}

\textit{Ученики}: Сечение куба не может быть $n$-угольником, где $n$ не больше шести.

\textit{Учитель}: А каким будет ограничение снизу?

\textit{Ученики}: Простейшей фигурой является треугольник, и он может  быть сечением куба, поэтому n от трех до шести.

\textit{Учитель}: А верно ли это для другого многогранника?

\textit{Ученик 3}: По определению сечения многогранника, сечением является многоугольник, вершины которого лежат на ребрах куба, а стороны целиком на гранях многогранника.

\textit{Ученик 2}: То есть, число граней многогранника есть верхняя граница числа возможных сечений.

\textit{Ученик 1}: Значит, в общем случае, сечением многогранника может быть n-угольник, где n не меньше трех и не больше числа граней многогранника.

\textit{Учитель}: Замечательно, мы пришли к доказательству теоремы.

\textit{Представленная задача дает возможность не просто изучить сечения куба, а организовать самостоятельное исследование учеников при модерации учителя с помощью использования метода неосократического диалога. В виду нехватки времени на уроке можно воспроизводить только часть этой беседы, любой опыт самостоятельных рассуждений будет полезен для развития школьников. Эту задачу можно рассматривать на обычных уроках, направленных на практическую работу с сечениями, или на факультативах, где у учащихся будет больше времени для рассуждений и рассмотрения всех выдвигаемых гипотез.}

\textbf{Пример 2.}

\textit{Следующая задача является примером классического задания на построение сечения куба с необходимостью проведения рассчетов для опредления расположения секущей плоскости в пространстве. Задача имеет несколько решений, из которых два рассмотрены в рамках несократического диалога. Если ученики выбрали один путь, нужно дать им провести рассуждения полностью, прийти к результату, и уже после обратить внимание на возможность другого решения.}

\textbf{Задача:} Дан куб $ABCDA_1B_1C_1D_1$ с точкой K, которая лежит на диагонали $AC_1$ и делит её в отношении $AK:KC_1 = 1:2$. Постройте сечение куба плоскостью, проходящей через точку K перпендикулярно этой диагонали.

\vspace{0.3cm}
\textit{Учитель}: Сегодня мы решаем с вами такую задачу. Что нам дано для построения сечения?

\begin{center}
\hspace{-2cm}
    \begin{minipage}{0.3\textwidth}
        \centering
        \drawCubeTaskOne[]{5} 
        % \vspace{-0.3cm}
        % \hspace{-1cm}
        % \caption{}
    \end{minipage}
    \capt{1}
\end{center}

\textit{Ученики}: Нам дана прямая, которой должно быть перпендикулярно наше сечение, и конкретная точка на этой прямой, через которую это сечение должно проходить. 

\textit{Учитель}: Что ещё нужно для построения искомого сечения?

\textit{Ставится общий неоднозначный вопрос, от которого зависит путь решения задачи.}

\textit{Ученик 1}: Для построения плоскости, перпендикулярной прямой, можно найти два пересекающихся перпендикуляра к этой прямой.

\textit{Ученик 2}: Для начала можно найти этот перпендикуляр, который лежит в плоскость $AA_1C_1C$, поскольку в ней полностью лежит $AC_1$ перпендикуляр к нашей искомой плоскости.

\textit{Учитель}: Хорошо, как найти в ней нужный нам перпендикуляр?

\textit{Ученики}: Сами проведем перпендикуляр к диагонали через точку K и пересечем его с AC в точке O. Точка O в нижней грани куба также будет точкой искомого сечения.

\begin{center}
\hspace{-3.9cm}
    \begin{minipage}{0.3\textwidth}
        \centering
        \drawCubeTaskOnePTwo[]{5} 
        % \vspace{-0.3cm}
        % \hspace{-1cm}
        % \caption{}
    \end{minipage}
    \capt{2}
\end{center}

\textit{Учитель}: А другие стороны прямоугольника $AA_1C_1C$ перпендикуляр пересечет?

\textit{Ученик 1}: Да, он будет пересекать отрезок $A_1C_1$, пусть, в точке $M$. 

\textit{Ложное суждение, после его высказывания необходимо, чтобы учащиеся пришли к противоречию и отвергли это предположение, высказав новую гипотезу.}

\begin{center}
\hspace{-3.89cm}
    \begin{minipage}{0.3\textwidth}
        \centering
        \drawCubeTaskOnePThree[]{5} 
        % \vspace{-0.3cm}
        % \hspace{-1cm}
        % \caption{}
    \end{minipage}
    \capt{2}
\end{center}

\textit{Учитель}: Как располагаются точки $O$ и $M$ на ребрах куба?

\textit{Ученик 1}: Мы можем вычислить отношения отрезков, на которые эти точки делят ребра. Так как $AC_1$ перпендикулярна $MO$, то треугольники $AKO$ и $ACC_1$ подобны по первому признаку подобия треугольников.

\textit{Ученик 2}: Пусть сторона куба равна $a$ . Тогда, так как $$AK:AC=\frac{a\sqrt{3}}{3} \cdot \frac{1}{a\sqrt{2}}=\frac{1}{\sqrt{6}}$$.

\textit{Ученик 3}: 
Значит, $\dfrac{AO}{AC_1}=\dfrac{1}{\sqrt{6}}$. Следовательно, $AP=\dfrac{a\sqrt{2}}{2}$ 

\textit{Ученик 1}: То есть, точка $O$ — середина $AC$!

\textit{Учитель}: Хорошо, это важный результат. А что там с точкой $M$?

\textit{В этот момент нужно зафиксировать правильный вывод, полученный в ходе рассмотрения неверного рисунка, чтобы, получив противоречие с расположением точки $M$, ученики не отвергли и вывод о том, что точка $O$ — середина отрезка $AC$.}

\textit{Ученик 1}: В этом случае можно также применить подобие треугольников. Треугольники $AKO$ и $C_1KM$ тоже подобны по первому признаку.

\textit{Ученик 3}: Подобны с коэффициентов подобия $\dfrac{1}{2}$.

\textit{Ученик 2}: Тогда длина отрезка $MC$ в два раза больше длины отрезка $AO$. Значит, $MC_1=A_1C_1$

\textit{Ученик 1}: То есть, точки $A_1$ и $M$ совпадают.

\textit{Учитель}: Мы пришли с вами к противоречию с предыдущим нашим предположением. Но сейчас точно определили точки пересечения перпендикуляра к $AC_1$ в точке $K$ с гранями куба. Что нам теперь необходимо сделать, чтобы получить искомое сечение?

\textit{Ученики}: Поскольку плоскость относительно перпендикуляра задается двумя пересекающимися прямыми, нам нужно найти вторую прямую, перпендикулярную исходной.

\textit{Учитель}: Хорошо, какой путь для ее нахождения вы видите?

\textit{Ученики}: Поскольку перпендикуляр к плоскости является диагональю куба, имеет смысл рассмотреть теорему о трех перпендикулярах, поскольку у нас уже есть проекция – это диагональ основания куба.

\textit{Учитель}: Так, и какая же это прямая?

\textit{Ученик 1}: По условию теоремы о трех перпендикулярах, нам нужна прямая, проходящая через точку $A$ и перпендикулярная $AC$. В плоскости нижней грани $BD$ и $AC$ перпендикулярны как диагонали, но BD не проходит через точку $A$.

\textit{Ученик 2}: Можно рассмотреть среднюю линию треугольника $ACC_1$, она параллельна $AC_1$, и её проекция – $OC$ перпендикулярна $BD$. Тогда по теореме о трёх перпендикулярах наклонная $OC$ перпендикулярна $BD$. 

\textit{Ученик 3}: А углы между скрещивающимися прямыми определяются с помощью пересекающихся параллельных им прямых, а значит $AC_1$ перпендикулярна $BD$.

\textit{Ученик 1}: Тогда мы уже имеем два перпендикуляра к $AC_1$: $A_1O$ и $BD$. Можно построить сечением достаточно только провести отрезки $A_1D$ и $A_1B$. То есть, треугольник $A_1BD$ – искомое сечение.

\textit{Рассмотрение различных способов решения одной и той же задачи полезно для развитие аналитических способностей учащихся и их математической грамотности.}

\textit{Учитель}: Так, выходит мы решили задачу, но можно ли было решить эту задачу без доказательства того, что точка $A_1$ будет точкой нашего сечения?

\textit{Ученик 3}: В этом случае мы знаем то, что перпендикуляр к $AC_1$ из точки K проходит через точку пересечения диагоналей квадрата $ABCD$. А значит, все также можно доказать, что $BD$ принадлежит секущей плоскости.

\textit{Ученик 2}: Тогда нужно найти еще одну прямую, перпендикулярную $AC_1$ и пересекающую $BD$. 

\textit{Ученик 1}: Снова применим теорему о трёх перпендикулярах. Отрезки $AB_1$, $AD_1$, $DC_1$ и $BC_1$ являются проекциями диагонали $AC_1$. Тогда отрезки $A_1B$, $A_1D$, $CD_1$ и $B_1C$ перпендикулярны $AC_1$

\textit{Ученик 3}: Но нам подходят только отрезки $A_1B$ и $A_1D$, так как только они пересекают $BD$. Собственно, эти три отрезка и образуют искомое сечение – треугольник $A_1BD$

\textit{Учитель}: Абсолютно верно, теперь мы получили еще одно решение этой задачи.

\vspace{0.4cm}

\textbf{Пример 3.}

\textit{В качестве третьего примера рассмотрим стреометрическую задачу из ЕГЭ по профильной математике. Так как экзамен, очевидно, требует умения самостоятельно рассуждать и решать сложные задачи, метод неосократического диалога позволяет развить критическое мышление, умение анализировать и задавать вопросы как собеседникам, так и самому себе. Таким образом внешний диалог с течением времени перейдет во внутренний.}

\textbf{Задача:} В правильной треугольной призме $ABCA_1B_1C_1$ сторона основания $AB $ равна 3, а боковое ребро $AA_1$ равно $\sqrt{2}$. На ребрах $AB$, $A_1B_1$ и $B_1C_1$ отмечены точки $M$, $N$ и $K$ соответственно, причем $AM=B_1N$=$C_1K=1$.

\textit а) Пусть $L$ – точка пересечения плоскости $MNK$ с ребром $AC$. Докажите, что $MNKL$ – квадрат.

\textit б) Найдите площадь сечения призмы плоскостью $MNK$.

\vspace{0.4cm}

\textit{Учитель}: Добрый день. Сейчас мы с вами будем решать стереометрическую задачу из ЕГЭ по профильной математике. Прочитайте внимательно условие. С чего начнем работу с задачей?

\textit{Ученики}: Нужно составить рисунок.

\textit{Учитель}: Как стоит расположить призму в пространстве?

\textit{Ученик 1}: Удобно будет, если мы будем видеть секущую плоскость как бы «спереди». Поэтому можно сделать такой рисунок (Рис. 1).

\textit{Этот ученик идёт к доске и делает рисунок.}

\begin{center}
\hspace{-3.8cm}
    \begin{minipage}{0.3\textwidth}
        \centering
        \drawTetrahedron[]{5} 
        % \vspace{-0.3cm}
        % \hspace{-1cm}
        % \caption{}
    \end{minipage}
    \capt{1}
\end{center}

\textit{Ученик 2}: Нам даны три точки плоскости, можем построить сечение.

\textit{Ученик 3}: Да, во-первых, проведем отрезки $KN$ и $NM$. Плоскость пересекает параллельные плоскости по параллельным прямым. Поэтому можно в грани $ABC$ провести прямую через точку $M$ параллельную $KN$.  

Учитель: Эта прямая будет пересекать ещё какую-нибудь сторону нижнего основания?

\textit{Ученик 3}: Да, пусть она пересекает $BC$ в точке $P$ (Рис. 2). 

\textit{Ошибка в расположении прямой.}

\begin{center}
\hspace{-3.8cm}
    \begin{minipage}{0.3\textwidth}
        \centering
        \drawTetrahedronPTwo[]{5} 
        % \vspace{-0.3cm}
        % \hspace{-1cm}
        % \caption{}
    \end{minipage}
    \capt{2}
\end{center}

\textit{Учитель}: Как относительно друг друга располагаются прямые $MP$ и $NK$?

\textit{Ученик 2}: Они должны быть параллельны.

\textit{Ученик 1}: Тогда треугольники $MPB$ и $NKB_1$ подобны.

\textit{Ученик 3}: Да, и длина отрезка $MB$ в два раза больше длины $NB_1$.

\textit{Ученик 2}: Длина $BP$ в два раза больше длины $B_1K$, то есть $BP=4$.

\textit{Ученик 1}: Но ведь $BC=4$.

\textit{Учитель}: Какое тогда предположение оказалось неверным?

\textit{Ученик 2}: Точка $P$ должна принадлежать ребру $AC$, чтобы $MP$ была параллельна $NK$. 

\textit{Ученик 3}: Тогда для соответствия условию задачи, её следует назвать точкой $L$ (Рис. 3).


\begin{center}
\hspace{-3.8cm}
    \begin{minipage}{0.3\textwidth}
        \centering
        \drawTetrahedronPThree[]{5} 
        % \vspace{-0.3cm}
        % \hspace{-1cm}
        % \caption{}
    \end{minipage}
    \capt{3}
\end{center}

\textit{Учитель}: А что необходимо, чтобы ответить на первый вопрос задачи?

\textit{Ученик 1}: Четырехугольник MNKL у нас построен. Чтобы доказать, что он квадрат, можно найти его стороны.

\textit{Ученик 3}: Так как $ML$ параллельна $NK$, и $AM=NB_1$, то треугольники $AML$ и $B_1NK$ равны.

\textit{Ученик 1}: Тогда по теореме косинусов следует, что $NK=ML=\sqrt{3}$.

\textit{Ученик 2}: А ещё, по теореме обратной теореме Пифагора, эти треугольники прямоугольные.

\textit{Учитель}: Вы доказали, что $MNKL$ является чем?

\textit{Ученик 3}: Если две стороны равны и параллельны, то это параллелограмм.

\textit{Доказательство того, что $MNKL$ – квадрат, способствует также систематизации знаний о видах четырехугольников. Последовательные рассуждения позволяют вспомнить и дополнительно закрепить признаки разных четырехугольников.}

\textit{Учитель}: А как найти длины других сторон?

\textit{Ученик 2}: $MN$ – сторона прямоугольной трапеции в боковой грани, поэтому, опустив высоту $MH$ из точки $M$ на $A_1N$, получим прямоугольный треугольник $
$.

\textit{Ученик 1}: Тогда, по теореме Пифагора, $MN=\sqrt{3}$, а значит и $LK=\sqrt{3}$.

\textit{Ученик 3}: То есть, мы доказали, что $MNKL$ – ромб, осталось доказать, что все его углы прямые.

\textit{Учитель}: Как же это лучше сделать?

\textit{Ученик 2}: Можно найти диагонали ромба. Если они равны, то это квадрат.

\textit{Ученик 1}: В этой задаче хорошо работает теорема Пифагора. Можно провести через точки $N$ и $K$ перпендикуляры к плоскости нижнего основания, пусть это будут $NF$ и $KS$.

\textit{Ученик 3}: Тогда треугольники $LAF$ и $MBS$ равные, равносторонние треугольники. Значит, $FL=MQ=2$.

\textit{Ученик 2}: И снова, по теореме Пифагора, $NL=KM=\sqrt{6}$. Значит, Диагонали ромба равны, то есть, $MNKL$ – квадрат. Что и требовалось доказать.

\textit{Учитель}: Обязательно ли было находить длину отрезков $NL$ и $KM$?

\textit{Обращаем внимание на необходимые и избыточные выводы для доказательства утверждения.}

\textit{Ученик 1}: Нет, по двум катетам, треугольники $MSK$ и $LFN$ равны, следовательно, $NL$ и $KM$ равны. И опять же, $MNKL$ – квадрат.

\textit{Учитель}: Теперь переходим ко второму пункту. Что необходимо сделать первым делом?

\textit{Ученик 3}: Нам нужно найти площадь площадь сечения, а сечение у нас квадрат, тогда его площадь равна 3.

\textit{Ученик 2}: Но $MNKL$ – это неполное сечение. Нужно ещё получить точку на ребре $CC_1$. Для этого воспользуемся методом следов. Пусть FL пересекает $BC$ в точке $P$, и пусть $PK$ пересекает $CC_1$ в точке $E$. 

\textit{Ученики научились сами искать противоречия в своих рассуждениях и вести диалог, что обеспечивает их большую самостоятельсность. Поэтому, если ученики научатся работать в рамках неосократического метода, у учителя появляется возможность отдать инициативу в руки учеников, сохраняя за собой возможность их направлять и регулировать течение беседы.}


\begin{center}
\hspace{-3.8cm}
    \begin{minipage}{0.3\textwidth}
        \centering
        \drawTetrahedronPFour[]{5} 
        % \vspace{-0.3cm}
        % \hspace{-1cm}
        % \caption{}
    \end{minipage}
    \capt{4}
\end{center}

\textit{Ученик 3}: Правда. Тогда, чтобы найти площадь сечения, можно найти отдельно площадь квадрата $MNKL$ и треугольника $KEL$, сумма этих площадей и будет искомой.

\textit{Ученик 1}: Площадь квадрата мы уже знаем, как найти площадь треугольника $KEL$? 

\textit {Ученики сами задают друг другу вопросы в процессе рассуждения.}

\textit{Ученик 2}: Можем, по теореме Пифагора, найти $KE$ и $EL$. А потом, например, по формуле Герона найти площадь треугольника.

\textit{Учитель}:  От чего зависят длины этих отрезков?

\textit{Ученик 1}: От положения точки E. Рассмотрим треугольники $EC_1K$ и ECP, они равны по катету и острому углу. Тогда отрезки $C_1E$ и EC равны, то есть точка E – середина ребра $CC_1$.  $C_1E$=$EC$=$\sqrt{2}$/2.

\textit{Ученик 2}: Тогда в треугольнике LEK длины сторон такие: $\sqrt{3}$, $\dfrac{\sqrt{3}}{2}$, $\frac {\sqrt{3}}{2}$. А они образуют пифагорову тройку. Тогда можно найти площадь этого треугольника как полупроизведение катетов. Значит, площадь треугольника $LEK$ равна $\dfrac{3}{4}$.

\textit{Ученик 3}: И, следовательно, общая площадь сечения равна $3\dfrac{3}{4}$ .

Учитель: Как ещё вы предлагали найти площадь треугольника $LEK$?

\textit{Ученик 1}: Предлагали найти с помощью формулы Герона, но в этом треугольнике длины сторон выражаются через квадратные корни, считать полупериметры было бы не удобно. Не каждый путь решения удобен для конкретных задач.

\textit{Учитель}: Замечательно. Мы с Вами полностью решили задачу из экзамена.

\textit{Таким образом, метод неосократического диалога позволяет решать на уроке не только исследовательские задачи, но и классические стереометрические задачи и задачи из экзамена, ведь умение рассуждать, анализировать и критически мыслить необходимо вне зависимости от конкретной деятельности человека.}










\begin{thebibliography}{}
\addcontentsline{toc}{section}{Список литературы }
\bibitem{} Дзюба, М. В. Метод неосократического диалога в классах с углубленным изучением математики при решении стереометрических задач по теме "Угол между скрещивающимися прямыми" / М. В. Дзюба // Современные проблемы математики и математического образования : Сборник научных трудов международной научной конференции, Санкт-Петербург, 16–18 апреля 2024 года. – Санкт-Петербург: Российский государственный педагогический университет им. А. И. Герцена, 2024. – С. 118-123. – EDN DVNBBG
\bibitem{} Иванова, О. Э. Сократический диалог как обучение совместному решению проблем / О. Э. Иванова, Т. Г. Точилкина // Азимут научных исследований: педагогика и психология. – 2017. – Т. 6, № 3(20). – С. 108-11. – EDN ZISRMJ.
\bibitem{} Мельникова, Е. Л. Проблемный урок, или Как открывать знания с учениками : Пособие для учителя / Е. Л. Мельникова. – Издание второе, стереотипное. – Москва : АПК и ПРО, 2006. – 168 с. – ISBN 5-8429-0080-7. – EDN ZBDXGX.
\bibitem{} Павлова, М. А. Сократовский диалог как метод исследовательского обучения экспериментальной математике на занятиях кружка / М. А. Павлова, М. В. Шабанова // Ярославский педагогический вестник. – 2015. – № 5. – С. 80-85. – EDN UZEYER.
\bibitem{}Смирнов, В. А. Задачи на распознавание сечений многогранников / В. А. Смирнов, И. М. Смирнова // Математика в школе. – 2019. – № 2. – С. 11-17. – EDN APCKWD.
\bibitem{}Сухорукова, Е. В. Сопровождение исследовательских проектов школьников / Е. В. Сухорукова // Актуальные проблемы прикладной математики, информатики и механики : Сборник трудов Международной научной конференции, Воронеж, 02–04 декабря 2024 года. – Воронеж: Научно-исследовательские публикации, 2025. – С. 1270-1273. – EDN TQBJQK.
\bibitem{}Фролова, М. В. Использование метода неосократического диалога как способа формирования критического мышления при обучении стереометрии учащихся специализированных классов / М. В. Фролова // Современные проблемы математики и математического образования : Сборник научных статей Международной научной конференции: к 225-летию Герценовского университета, Санкт-Петербург, 04–06 июня 2022 года / Под редакцией В.В. Орлова и М.Я. Якубсона. – Санкт-Петербург: Российский государственный педагогический университет им. А. И. Герцена, 2022. – С. 113-118. – EDN CRLEJA.
\bibitem{}Фролова, М. В. Особенности использования метода неосократического диалога при обучении стереометрии / М. В. Фролова // Современные проблемы математики и математического образования : Международная научная конференция «78 Герценовские чтения», Санкт-Петербург, 15–17 апреля 2025 года. – Санкт-Петербург: Издательско-полиграфическая ассоциация высших учебных заведений, 2025. – С. 144-149. – EDN UHPKTO.
\bibitem{}Фролова, М. В. Сократический диалог как способ формирования навыков рассуждения на уроках стереометрии в классах с углубленным изучением математики / М. В. Фролова // Классическая и современная геометрия : материалы международной конференции, посвященной 100-летию со дня рождения Л. С. Атанасяна, Москва, 01–04 ноября 2021 года / Московский педагогический государственный университет. – Москва: Московский педагогический государственный университет, 2021. – С. 149-150. – EDN LRKTTC.


\end{thebibliography}
\end{document}
